\documentclass[preprint,12pt]{elsarticle}


%% The `ecrc' package must be called to make the CRC functionality available
%\usepackage{ecrc}

%% set the volume if you know. Otherwise `00'
%\volume{00}

%% set the starting page if not 1
%\firstpage{1}

%% Give the name of the journal
%\journalname{Expert Systems With Applications}

%% Give the author list to appear in the running head
%% Example \runauth{C.V. Radhakrishnan et al.}
%\runauth{}

%% The choice of journal logo is determined by the \jid and \jnltitlelogo commands.
%% A user-supplied logo with the name <\jid>logo.pdf will be inserted if present.
%% e.g. if \jid{yspmi} the system will look for a file yspmilogo.pdf
%% Otherwise the content of \jnltitlelogo will be set between horizontal lines as a default logo

%% Give the abbreviation of the Journal.  Contact the journal editorial office if in any doubt
%\jid{eswa}

%% Give a short journal name for the dummy logo (if needed)
%\jnltitlelogo{ESWA Logo}

%% Provide the copyright line to appear in the abstract
%% Usage:
%   \CopyrightLine[<text-before-year>]{<year>}{<restt-of-the-copyright-text>}
%   \CopyrightLine[Crown copyright]{2011}{Published by Elsevier Ltd.}
%   \CopyrightLine{2011}{Elsevier Ltd. All rights reserved}
%\CopyrightLine{2013}{Published by Elsevier Ltd.}



%\usepackage{llncsdoc}
\usepackage[figuresright]{rotating}
%\usepackage{makeidx}  % allows for indexgeneration
\usepackage{graphicx}
\usepackage[T1]{fontenc}
\usepackage[english]{babel}
\usepackage[utf8]{inputenc}
% \usepackage{multirow}

\usepackage{url}
\usepackage{rotating}

%%%Math
\usepackage{latexsym}
 \usepackage{amsmath}
% \usepackage{amssymb}
% \usepackage{amsthm}
%\usepackage{eurosans}

\usepackage{eurosym}

\usepackage{longtable}

\usepackage{listings}

\usepackage{color}
\usepackage{textcomp}


\definecolor{gray}{gray}{0.5}
\definecolor{green}{rgb}{0,0.5,0}
% 
% \usepackage{inconsolata}



\begin{document}


\begin{frontmatter}

%% Title, authors and addresses

%% use the tnoteref command within \title for footnotes;
%% use the tnotetext command for the associated footnote;
%% use the fnref command within \author or \address for footnotes;
%% use the fntext command for the associated footnote;
%% use the corref command within \author for corresponding author footnotes;
%% use the cortext command for the associated footnote;
%% use the ead command for the email address,
%% and the form \ead[url] for the home page:
%%
%% \title{Title\tnoteref{label1}}
%% \tnotetext[label1]{}
%% \author{Name\corref{cor1}\fnref{label2}}
%% \ead{email address}
%% \ead[url]{home page}
%% \fntext[label2]{}
%% \cortext[cor1]{}
%% \address{Address\fnref{label3}}
%% \fntext[label3]{}

%\dochead{}
%% Use \dochead if there is an article header, e.g. \dochead{Short communication}
%% \dochead can also be used to include a conference title, if directed by the editors
%% e.g. \dochead{17th International Conference on Dynamical Processes in Excited States of Solids}


\title{Semantic-based QoS management in Cloud Systems: Current Status and Future Challenges}


%% use optional labels to link authors explicitly to addresses:
% \author[label1]{Jose María Alvarez-Rodríguez\corref{cor1}}
% \address[label1]{The South East European Research Center, Thessaloniki, Greece.}
% \ead{jmalvarez@seerc.org}
% \ead[url]{http://www.seerc.org}
% 
% \author[label2]{José Emilio Labra-Gayo}
% \address[label2]{WESO Research Group, Department of Computer Science, University of Oviedo, 33007, Oviedo, Spain.}
% \ead{labra@uniovi.es}
% 
% \author[label3]{Alejandro Rodríguez-González}
% \address[label3]{Bioinformatics at Centre for Plant Biotechnology and Genomics UPM-INIA, Polytechnic University of Madrid, Madrid, Spain.}
% \ead{alejandro.rodriguezg@upm.es}
% 
% \author[label4]{Patricia Ordoñez De Pablos}
% \address[label4]{WESO Research Group, Department of Business Administration, University of Oviedo, 33007, Oviedo, Spain.}
% \ead{patriop@uniovi.es}





\author{}

\address{}

\begin{abstract}
The concept of Cloud Computing and Service Oriented Architectures have seen a 
dramatic increase of the amount of applications, services, management platforms, 
data, etc. gaining momentum in Information and Communication Technology and more 
specifically in the deployment of the Future Internet. However this explosion 
implies the necessity of new complex methods and techniques to deal with the 
vast heterogeneity of data sources, services or platforms. In this sense Quality 
of Service (QoS) seeks for providing an intelligent environment of 
self-management components based on domain knowledge in which both functional 
and nonfunctional properties of cloud components can be optimized in terms of 
cost, efficiency or reliability easing the transition to an advanced resource 
provisioning process. On the other hand, semantics and ontologies have emerged 
as part of the Artificial Intelligence to afford a common and standard data 
model that ease the interoperability, integration and monitorization of 
knowledge-based systems. Furthermore the Linked Data initiative as practical 
view of the Semantic Web has posed the baseline technology to easily integrate, 
enrich and consume data in a distributed system. Taking into account the 
necessity of an intelligent system to manage QoS in Cloud Systems and the 
emerging application of semantics, ontologies and Linked Data in different 
domains, this paper reviews the main approaches for semantic-based QoS 
management as well as the principal methods, techniques and standards for 
processing and exploiting diverse data providing advanced real-time monitoring 
services for QoS. A discussion of existing efforts and challenges are also 
provided to suggest future directions.
\end{abstract}

\begin{keyword}
%% keywords here, in the form: keyword \sep keyword
cloud systems \sep quality of service \sep  service oriented architectures \sep  semantics \sep  ontologies \sep  linked data \sep  sensor data \sep  big data 
%% PACS codes here, in the form: \PACS code \sep code

%% MSC codes here, in the form: \MSC code \sep code
%% or \MSC[2008] code \sep code (2000 is the default)

\end{keyword}


\end{frontmatter}

\section{Introduction}
\section{Literature Review}\label{sect:related-work}

\section{Evaluation}\label{sect:evaluation}

\section{Conclusions and Future work}\label{sect:conclusions}

\nocite{*}
\bibliographystyle{plain}
% %\bibliographystyle{unsrt}
% % %\bibliographystyle{acm}
\bibliography{cloud-references}
 % \renewcommand{\bibname}{References}


\end{document}
