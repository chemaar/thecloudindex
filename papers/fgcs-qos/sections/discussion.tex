Despite the growing interest in Cloud Computing [102, 103] and the hype of this 
new paradigm for the creation of new era of applications, services and data 
available through the Web, a real advanced cloud management environment is far 
from being fully developed. There are many open issues to be solved and 
technology to ease the transition from traditional developments and applications 
to a cloud-based environment is still under development. With regards to QoS, 
there are a lot of initiatives and efforts trying to model and manage functional 
and non-functional properties in an intelligent fashion. Nevertheless the lack 
of standards for unifying information and data is preventing the deployment of 
advanced techniques for QoS management. In the case of semantic technologies, 
works in different areas are emerging to solve interoperability and integration 
problems in distributed environments. More specifically, the creation of 
knowledge-based systems applying semantic-based techniques as stream reasoning 
and CEP are currently being developed to deal mainly with Big Data problems in the 
context of social media or e-Government. Following a list of particular comments 
to discuss are presented:
\begin{itemize}
 \item There are a big variety of QoS dimensions to ensure in a cloud system. 
The methods to ensure reliability, security and trust should be modeled and 
discovered in automatic ways. In this specific case it is also required to take 
into account user feedback to evaluate the real quality and trust of a service. 
Moreover, depending on the cloud layer, specific QoS characteristics should be 
defined to collect the requirements of each particular case. Currently QoS 
approaches are mainly focused on web service discovery and selection but new 
ranking methods and reactive control systems taking into QoS features should be 
deployed to provide an intelligent cloud infrastructure.  

 \item Information about a service consists in both static and dynamic data, the 
matchmaking of services according to these issues is a key-enabler for a real 
QoS management but it is not fully addressed by existing works.

 \item According to Table  the big variety of ontologies, OWL models, etc. that 
have been designed in recent years imply a tangled set of options that should be 
unified to provide an unique view of  what QoS should be. Apart from that the 
use of semantics is not clear, in some cases reasoning processes are used to 
discover services but others just define an ontology as a proposal to provide a 
formal model without any real application. A clear semantic-based architecture 
[104, 105] should be defined containing the adequate definitions [106].of 
functional and non-functional properties.

 \item With regards to Semantic Web and reasoning, there is a growing community 
trying to provide technology for supporting intelligent systems in the new Web 
of Data. As a consequence the necessity of dealing with Big Data problems and 
data coming from different sources is stimulating the creation of new approaches 
to reuse existing technology such as Apache Hadoop in the context of querying 
large datasets. Therefore the main application of the Semantic Web principles 
lies in the unification of data and the execution of reasoning processes to 
validate data and infer new facts. Nevertheless, the existing logic formalisms 
available in OWL such as DL, FOL, F-Logic, etc. do not seem to be a solution to 
tackle the challenge of modeling dynamic domains that is why some works [107] 
regarding Continuous Semantics are emerging.

 \item A research area that is expected to play a key-role to enrich information 
is Linked Data. Currently this initiative has been successfully applied to 
information retrieval systems or in the creation of rich user interfaces. 
Nevertheless, the expectations of linking different datasets to enrich 
information are growing as a manner for delivering more intelligent services. To 
apply Linked Data in the Cloud Computing environment we should ensure that any 
resource to be monitored has an URI, its data is coded into RDF according to a 
formal model, an API or endpoint is accessible for fetching data using pulling 
pushing or triggering techniques and the methods for graph processing and 
reasoning are efficient and scalable under real-time constraints.
 
\end{itemize}

As main future challenge among others [108, 109], Semantics can help to increase 
the reliability in Cloud Computing providing the building blocks and models for 
an advanced QoS management. In the same way, Cloud Computing can help semantic 
technologies to be more scalable and flexible making use of the web as 
infrastructure to create large-scale data-intensive batch applications [110].

