% Since 
The vast amount of cloud infrastructures, platforms and services are 
becoming a major challenge for technicians and IT/Business managers. According 
to FIXME one of the next big things in this environmens lies in the creation 
of cloud brokers that can automatically fulfill the requirements of an application, user 
or organization in an automatic way. In this the accomplishment of QoS indicators, 
SLA aggremeents, ECA rules, etc. is consider a key-enabler to enable the creation 
of cloud brokers. The aim of these brokers is then the simplification and 
the transition to other cloud services helping to overcome some issues such as 
the recommendation of service (XaaS) providers according to a profile. Cloud brokering 
is consider as an accelerant to help organizations to overcome their initial lack 
of experience and to manage the different service levels of an organization.

Therefore a cloud service brokerage (CSB) FIXME: Gartner,is a kind of services intermediation 
``in which a company or other entity adds value to one or more (generally public or hybrid, but possibly private) 
cloud services on behalf of one or more consumers of that service''. Some of the capabilities of a 
CSB must include: governance, community management, service enrichment and deliver, distributed quality of service, 
analytics and operational intelligence and SaaS and custom SaaS (to support business processes). 
Although its benefiting rating is set as ``High'', the market penetration is just ``1\% to 5\% of target audience'' 
and its maturity is still in an ``Adolescent'' status. According to the ``MUST'' capabilities of a CSB it 
seems that a QoS-driven CSB can really help organizations to improve or deploy the business models 
using a cloud infrastructure. Nevertheless a right deployment of a CSB will depend on its capability 
to process diverse data and perform analytics processes in an adequate fashion. Furthermore 
the use CSB is supposed to be used for both technicians and business users to make their 
own decisions in different aspects: 1) select a service provider or an API and 2) select 
the ``best'' cloud infrastructure for organization business.

For instance let's suppose that a developer needs to implement a new mobile application 
with a geocoding service and a response time in terms of milliseconds. The selection of this 
API is not a mere process of ensuring the functionality but to track 
and assess ``developer's quality criteria''. In this sense, a CBS based on QoS management can 
deal with some of the question that can arise such as:
\begin{itemize}
 \item Which is the ``best'' API for a geocoding service? (more than 54 existing geocoding APIs) 
 \item How can the developer compare (ranking of) different providers? 
 \item How can the developer track the quality (response time) of the selected service?
 \item \ldots
\end{itemize}

Now let's turn this simple example into a generic ``template'' for a CBS. Let's suppose that an agent 
needs to (create|deploy|implement|move) a new action re-using an existing service (XaaS) 
under a certain set of key performance indicators (KPIs).
\begin{itemize}
 \item Which is the ``best'' provider for the service (XaaS)? 
 \item How can the agent compare (ranking of) different providers? 
 \item How can the agent track the quality (KPIs) of the selected service (XaaS)?
  \item \ldots
\end{itemize}

In this scenario a CBS can clearly help ``agents'' to select the appropiate service and provider 
but it is necessary to enable of way of expressing and computing the ``best'' provider, comparing 
existing ones and tracking the selected one according to ``my'' quality restrictions. Due to 
the aforementioned tangled environment it requires the use of the righ methods to model user intentions and 
provider capabilities, to integrate data and compute...FIXME. As main conclusion a CBS supported by QoS indicators 
can be the next big thing in a Cloud Computing environment but the use of semantics and 
distributed processing techniques of datastreams must be considered to enable the properly 
exploitation on time and help agents to make decisions.








