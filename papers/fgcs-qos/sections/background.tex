This section presents a brief summary of the background concepts reviewed in this paper: 1) Cloud Computing and QoS; 2) Semantic Technologies 
and 3) Big Data, with the aim of having a better understanding of the requirements of QoS in Cloud Computing.

% Finally, the Semantic Web initiative tries to elevate the meaning of web information resources through common and shared data models 
% with the objective of improving the interoperability and integration in service oriented architectures.  In this sense two major efforts are being carried out: 1) the Linked Data~\cite{Berners-Lee-2006,Heath_Bizer_2011} 
% initiative that proposes a set of principles for publishing information and data using the Resource Description Framework (RDF) to ease the creation of 
% the new Web of Data realm. Thus data can be easily shared, exchanged and linked to other datasets through URIs. Existing works are focused on publishing and 
% consuming data via the SPARQL language. The main aim of this effort is to create a new environment of added-value services that can encourage and improve B2B (Business to Business), 
% B2C (Business to Client) or A2A (Administration to Administration) relationships. In the case of Cloud Computing and Linked Data there is a growing interest to process large datasets in the 
% cloud and works with regards to distributed query processing are emerging and 2) the design and development of logic formalisms and knowledge-based systems using ontologies 
% in OWL (Ontology Web Language). In this sense new expert systems are arisen to tackle existing problems in medical reasoning, analysis of social media, etc. in which data heterogeneities, 
% lack of standard knowledge representation and interoperability problems are common factors. Finally the use of semantic technologies in service oriented architectures has been widely 
% studied in the Semantic Web Services area and some interesting works [15] such as the W3C Recommendation Semantic Annotations for WSDL (SAWDL), the WSMO ontology, the WSML language, 
% the OWL-S language for describing web services, the WSMX execution environment, etc. can be found. Although Semantic Web Services were very promising and some advances 
% in discovery or selection processes were made but we consider these efforts are out of the scope of this review.

Since the core concepts of this review are defined, it is clear that semantic web technologies offer a standard and unified way for representing information and data coming from cloud-based applications. 
QoS management processes can take advantage of this situation building expert systems that exploit this data to support the abovementioned five key-characteristics of 
Cloud Computing providing an intelligent and flexible environment for self-managed applications.
