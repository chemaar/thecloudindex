Cloud Computing represents the next natural step in the evolution of on-demand services and applications. Several definitions have been made but the description~\cite{mell2011nist} 
provided by the NIST institute has reached a major consensus:  ``A large-scale distributed computing paradigm that is driven by economies of scale, in which a pool of 
abstracted, virtualized, dynamically-scalable, managed computing power, storage, platforms, and services are delivered on demand to external customers over the Internet''. 
The NIST institute has also defined~\cite{mell2011nist,Garcia-Sanchez:2010:ASS:1852403.1852409}: 
\begin{enumerate}
 \item Five key characteristics: on-demand self-service, ubiquitous network access, location independent resource pooling, rapid elasticity and pay per use.
 \item Three service models: Software-as-a-Service (SaaS), Platform-as-a-Service (PaaS) and Infrastructure-as-a-Service (IaaS).
 \item Four development models: private, community, public and hybrid clouds. 
\end{enumerate}

These basic concepts in a cloud environment lead us to consider that QoS is a key-enabler of the five essential characteristics identified by 
the NIST institute and it is closely related to the concepts of Autonomic and Utility Computing~\cite{Huebscher:2008:SAC:1380584.1380585}. As a consequence the QoS management 
is clearly a key-enabler of cloud environments and it must play a major role in the Future Internet.

On the other hand, the ITUT-T Recommendation E.800 defines QoS as ``collective effect of service performance that determines the degree of satisfaction by a user of the service''. 
Thus QoS data is a key-enabler to design, identify and put in action SLAs. It should also influence software components and applications to ensure a 
reliable environment for executing services. Some open issues in QoS management emerge to extend this definition including reputation-based mechanisms 
for service selection or dynamic adaptation of resource provisioning. The application of QoS has been widely studied and applied~\cite{Conejero:2012:MSQ:2357487.2357591,Pedersen:2011:AMQ:2114495.2115542} 
in the web services and grid computing area and it is gaining momentum in the new Cloud Computing paradigm. In order to facilitate the QoS management in the cloud-environment 
some tools, called Cloud Management Platforms (CMPs) can be found to manage the different layers of cloud-based applications but the majority of them are 
now focused on the IaaS layer. The use of these platforms can help to manage the growing of cloud applications and ease the deployment and monitoring of services across 
public and private clouds. The six key capabilities~\cite{Kephart2012} that we should look for in a CMP are: simplify complexity, 
manage multiple clouds, build for the future, support the whole application lifecycle, self-management (set-it and forget-it) and manage/control costs. 
Apart from that most of the vendors offer cost forecasting, alert service and reporting to ensure an adequate intelligent management. 
Examples of CMPs can be found in RightScale, Enstratus, ScaleUp, Cloudability, Cloudyn, CloudExpress, MyGratitant or Cloudbees among others. 
Although these commercial products offer a very good option to manage cloud-based applications there is a lack of standardization and some 
QoS features cannot be managed. That is why OASIS just launched a CAMP TC~\cite{OASISCamp} to create and interoperable protocol that cloud 
implementers can use to package and deploy their applications. The idea is to provide a set of REST services, at the PaaS layer, to foster an ecosystem of 
common tools, plugins, libraries and frameworks, which will allow vendors to offer greater value-add. In the particular case of QoS, the use of standards to gather data 
from applications can improve the process of making decisions about resource provisioning or help in saving costs among others.

Finally, the Semantic Web initiative tries to elevate the meaning of web information resources through common and shared data models 
with the objective of improving the interoperability and integration in service oriented architectures.  In this sense two major efforts are being carried out: 1) the Linked Data~\cite{Berners-Lee-2006,Heath_Bizer_2011} 
initiative that proposes a set of principles for publishing information and data using the Resource Description Framework (RDF) to ease the creation of 
the new Web of Data realm. Thus data can be easily shared, exchanged and linked to other datasets through URIs. Existing works are focused on publishing and 
consuming data via the SPARQL language. The main aim of this effort is to create a new environment of added-value services that can encourage and improve B2B (Business to Business), 
B2C (Business to Client) or A2A (Administration to Administration) relationships. In the case of Cloud Computing and Linked Data there is a growing interest to process large datasets in the 
cloud and works with regards to distributed query processing are emerging and 2) the design and development of logic formalisms and knowledge-based systems using ontologies 
in OWL (Ontology Web Language). In this sense new expert systems are arisen to tackle existing problems in medical reasoning, analysis of social media, etc. in which data heterogeneities, 
lack of standard knowledge representation and interoperability problems are common factors. Finally the use of semantic technologies in service oriented architectures has been widely 
studied in the Semantic Web Services area and some interesting works [15] such as the W3C Recommendation Semantic Annotations for WSDL (SAWDL), the WSMO ontology, the WSML language, 
the OWL-S language for describing web services, the WSMX execution environment, etc. can be found. Although Semantic Web Services were very promising and some advances 
in discovery or selection processes were made but we consider these efforts are out of the scope of this review.

Since the core concepts of this review are defined, it is clear that semantic web technologies offer a standard and unified way for representing information and data coming from cloud-based applications. 
QoS management processes can take advantage of this situation building expert systems that exploit this data to support the abovementioned five key-characteristics of 
Cloud Computing providing an intelligent and flexible environment for self-managed applications.
