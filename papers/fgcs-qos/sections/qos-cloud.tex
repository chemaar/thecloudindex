Cloud Computing represents the next natural step in the evolution of on-demand services and applications. 
Several definitions have been made but the description~\cite{mell2011nist} provided by the NIST institute has reached a major consensus:  
``A large-scale distributed computing paradigm that is driven by economies of scale, in which a pool of 
abstracted, virtualized, dynamically-scalable, managed computing power, storage, platforms, and services are delivered on demand to external customers over the Internet''. 
The NIST institute has also defined~\cite{mell2011nist,Garcia-Sanchez:2010:ASS:1852403.1852409}: 
\begin{enumerate}
 \item Five key characteristics: on-demand self-service, ubiquitous network access, location independent resource pooling, rapid elasticity and pay per use.
 \item Three service models: Software-as-a-Service (SaaS), Platform-as-a-Service (PaaS) and Infrastructure-as-a-Service (IaaS).
 \item Four development models: private, community, public and hybrid clouds. 
\end{enumerate}

These basic concepts~\cite{mell2011nist} and usages~\cite{cloud-usage} in a cloud environment lead us to consider that QoS is a key-enabler of the five essential characteristics identified by 
the NIST institute and it is closely related to the concepts of Autonomic and Utility Computing~\cite{Huebscher:2008:SAC:1380584.1380585}. 
As a consequence the QoS management is clearly a key-enabler of cloud environments and it must play a major role in the Future Internet to 
afford, from a quality point of view, the implementation of the ``Measured Service'' concept, see Figure~\ref{fig:qos-intro}.

On the other hand, the ITUT-T Recommendation E.800 defines QoS as ``collective effect of service performance that determines the degree of 
satisfaction by a user of the service''. Thus QoS data is a key-enabler to design, identify and put in action SLAs. It should also influence 
software components and applications to ensure a reliable environment for executing services. Some open issues in 
QoS management emerge to extend this definition including reputation-based mechanisms for service selection or 
dynamic adaptation of resource provisioning. The application of QoS has been widely studied and 
applied~\cite{Conejero:2012:MSQ:2357487.2357591,Pedersen:2011:AMQ:2114495.2115542} to web services and grid computing areas and it is now 
gaining momentum in the new Cloud Computing paradigm. 

In order to facilitate the QoS management in the cloud-environment some tools, called Cloud Management Platforms (CMPs) can be found 
to manage the different layers of cloud-based applications but the majority of them are now focused on the IaaS layer. 
The use of these platforms can help to manage the growing of cloud applications and ease the deployment and monitoring of services across 
public and private clouds. The six key capabilities~\cite{Kephart2012} that we should look for in a CMP are: simplify complexity, 
manage multiple clouds, build for the future, support the whole application lifecycle, self-management (set-it and forget-it) and manage/control costs. 
In this sense OASIS just launched a CAMP TC~\cite{OASISCamp} to create and inter-operable protocol that cloud 
implementers can use to package and deploy their applications. The idea is to provide a set of REST services, at the PaaS layer, to foster an ecosystem of 
common tools, plug-ins, libraries and frameworks, which will allow vendors to offer greater value-add. In the particular case of QoS, the use of standards to gather data 
from applications can improve the process of making decisions about resource provisioning or help in saving costs among others. Nevertheless 
this specification is still in an early stage and its objectives are more focused on the management of cross-cloud applications than a 
real management from QoS point of view. Following main characteristics of the CAMP specification are presented:
\begin{itemize}
 \item It is a language, framework and platform neutral to manage in the same way Java, Ruby on Rails or Node.js applications.
 \item It only covers interactions between a cloud consumer and a provider~\cite{mell2011nist}.
 \item It supports the management of the entire lifecycle of the application not just the deployment of isolated components.
 \item A major objective of the specification is to provide an inter-operable environment. They are trying to keep simple as 
 possible but with the possibility of being extended by third-parties.
\end{itemize}

Although there is no a clear objective to support quality indicators it can be considered as a major effort to unify information exposed by providers and 
improve the creation of an integrated and inter-operable ecosystem in which existing cloud management application platforms such 
as RightScale, Enstratus, ScaleUp, Cloudability, Cloudyn, CloudExpress or MyGravitant can take advantage of implemented 
added-value services on the top of a common API. Although these commercial products offer a very good option 
to manage cloud-based applications there is a lack of standardization and some QoS features cannot be managed.  
Nevertheless these existing cloud management platforms present some interesting features and services for 
multi-cloud management that must be taken into account in the design of a dedicated cloud quality management platform such as:
\begin{itemize}
 \item Integrated management of cloud resources: compute, networking and storage.
 \item Organization: tagging capabilities or creation of profiles and views to group cloud resources.
 \item Accessibility and usability: monitoring (tracking and graphing custom metrics), dashboard or import capabilities.
 \item Custom services: creation of alerts using a particular set of indicators, cost forecasting or reporting.
\end{itemize}

%Policy making
On the other hand there is an interesting approach to manage cloud quality indicators using a policy-making 
perspective. In this sense public and private bodies are continuously seeking for new analytical tools and methods to 
assess, rank and compare their performance based on distinct indicators and dimensions with the objective of making 
some decision or developing a new policy. In this context the creation and use of quantitative indexes is 
a widely accepted practice that has been applied to various domains such as Bibliometrics and academic performance and 
quality (the Impact Factor by Thomson-Reuters, the H-index or the Shanghai and Webometrics rankings) the Web impact (the Webindex~\cite{webindexlod} 
by the Webfoundation) or Smart Cities (The European Smart Cities ranking) to name a few. Therefore policymakers as well as individuals are 
continuously evaluating quantitative measures to tackle or improve existing problems in different areas and 
support their decisions. Nevertheless the sheer mass of data now available is raising a new dynamic and challenging environment 
in which traditional tools are facing major problems to deal with data-sources diversity, structural issues or complex processes of estimation. 
According to some efforts such as the ``Policy-making $2.0$'' within the Cross-Over project~\footnote{\url{http://www.crossover-project.eu/}} 
that \textit{refers to a blend of emerging and fast developing technologies that enable better, more timely and more participated decision-making}, 
new paradigms and tools are required to take advantage of the existing environment (open data and big data) to design and estimate 
actions in this dynamic context according to requirements of transparency, standardization, adaptability and extensibility among 
others with the aim of providing new context-aware and added-value services such as visualization that 
can help a deepen and broaden understanding of the impact of a policy in a more fast and efficient way. 
As a consequence common features and requirements can be extracted from the existing situation out:
\begin{itemize}
 \item Data sources. Data and information is continuously being generated as observations from social networks, public and private institutions, NGOs, services and applications, etc. 
 creating a tangled environment of sources, formats and access protocols with a huge but restricted potential for exploitation. Nevertheless data processing, knowledge inferring, etc. are not mere processes 
 of gathering and analyzing, it is necessary to deal with semantic and syntactic issues, e.g. particular measurements and dimensions or name mismatches, 
 in order to enable a proper data/information re-use and knowledge generation.
 
 \item Structure. Quantitative indexes are usually defined (a mathematical model) by experts to aggregate several indicators (in a hierarchy structure) 
 in just one value to provide a measure of the impact or performance of some policy in a certain context. The structure of these indexes are 
 obviously subjected to change over time  to collect more information or adjust their composition and relationships (narrower/broader). 
 That is why technology should be able to afford adequate techniques to automatically populate new changes in an efficient way.
 
  \item Computation process. This feature refers to the calculation of the index. Observations are gathered from diverse data sources and aligned 
  to the index structure, commonly indicators, that are processed through various mathematical operators to generate a final index value. 
  Nevertheless the computation process is not always described neither open (any minor change can imply a long time for validation) implying that 
  cannot be easily replied for third-parties with other purposes, for instance research, preventing one 
  of the most wanted characteristics such as transparency. Furthermore it is necessary to ensure that the computation process 
  is sound and correct.

  \item Documentation. As the European project Cross-over has stated, new policy-making strategies go ahead of a simple and closed value and it is necessary to bring 
  new ways of exploiting data and information. Moreover the use of the Web as a dissemination channel represents a powerful environment in which 
  information should be available taking into account the multilingual and multicultural character of information. In this context documentation mechanisms 
  must necessarily cover all the aforementioned features to afford a detailed explanation of a quantitative index-based policy to both policymakers 
  and final users. However existing initiatives usually generates some kind of hand-made report which is not easy to keep up-to-date and deliver 
  to the long-tail of interested third-parties.
\end{itemize}

Following this perspective of creating a quantitative index the Cloud Computing community~\cite{Maiya:2012:QMC:2353730.2353862,DBLP:conf/quatic/KlemsBW12} 
and some of the big players have launched some relevant indexes such as:

\begin{itemize}
 \item The Service Measurement Index~\cite{smi} (SMI) by the Cloud Services Measurement Initiative Consortium (CSMIC) consortium. It is 
 based on a framework of critical key performance indicators (both business and technical) \textit{associated attributes, and measures 
 that provide a standardized method for measuring and comparing a business service regardless of whether that service is internally provided or sourced from an outside company for any cloud service (IaaS, PaaS, SaaS, etc.). It is designed to become a 
 standard method to help organizations measure cloud-based services based on their specific business and technology requirements.} As 
 an implementation of the SMI Index the ``Ranking Clouds''~\cite{DBLP:journals/fgcs/GargVB13} presents a framework to manage this index 
 and makes a ranking of different PaaS providers using the AHP method to weight some key performance indicators.
 
 \item The Cisco Global Cloud Index (GCI) by Cisco~\cite{cisco}. \textit{It is an ongoing effort to forecast the growth of global data 
 center and cloud-based IP traffic. The forecast includes trends associated with data center virtualization and cloud computing. }
 It also provides a visual representation of countries ranging  from ``Cloud Prepared'' to ``Cloud Emerging''.

 \item The CSC Cloud Usage index~\cite{csc}. It is an index created through a survey of more than ·$3500$ cloud computing users 
 in eight countries around the world. The survey is focused on capturing user information about outcomes and 
 experiences rather than predictions and intentions.

 \item The VMWare Cloud index~\cite{vmware}. It is an study conducted by Forrester Consulting and ITR in Japan. \textit{The 2012 study surveyed 
 approximately $6500$ senior IT practitioners across the APJ region in eleven countries / regions with the aim of establishing 
 top cloud drivers and concerns in the community.}

\end{itemize}

FIXME: some concluding remarks?

% 
% 
% \begin{sidewaystable}[!ht]
% \renewcommand{\arraystretch}{1.3}
% \tiny
% \begin{center}
% \begin{tabular}[c]{|p{2.5cm}|p{9cm}|p{1.5cm}|p{1.5cm}|p{1.5cm}|} 
% \hline
%   \textbf{KPI} &  \textbf{Definition}  &  \textbf{Broader} & \textbf{Defined by} & \textbf{Scope} \\\hline
%   Accountability&This component is used to measure the properties related to the service provider
% organization. These properties may be independent of the service being provided.&&CSMIC&$\ast$ \\ \hline
%   Agility&Indicates the impact of a service upon a client's ability to change direction, strategy, or tactics quickly and with minimal disruption.&&CSMIC&$\ast$ \\ \hline
%   Assurance&This category includes key attributes that indicate how likely it is that the service will be available as specified.&&CSMIC&$\ast$ \\ \hline 
%   Financial&The amount of money spent on the service by the client.&&CSMIC&$\ast$ \\ \hline 
%   Performance&This category covers the features and functions of the provided services.&&CSMIC&$\ast$ \\ \hline 
%   Security and Privacy&This category includes attributes that indicate the effectiveness of a service provider's controls on access to services, service data, and the physical facilities from which services are provided.&&CSMIC&$\ast$ \\ \hline 
%   Usability&The ease with which a service can be used.&&CSMIC&$\ast$ \\ \hline 
%   Auditability&The ability of a client to verify that the service provider is adhering to the standards, processes, and policies that they follow.&Accountability&CSMIC&$\ast$ \\ \hline 
%   Compliance&Standards, processes, and policies committed to by the service provider are followed.&Accountability&CSMIC&$\ast$ \\ \hline 
%   Contracting Experience&Indicators of client effort and satisfaction with the process of entering into the agreements required to use a service.&Accountability&CSMIC&$\ast$ \\ \hline 
%   Data Ownership&The Level of rights a client has over client data associated with a service.&Accountability&CSMIC&$\ast$ \\ \hline 
%   Ease of Doing Business&Client satisfaction with the ability to do business with a service provider.&Accountability&CSMIC&$\ast$ \\ \hline 
%   Ease of Doing Business&The processes used by the service provider to manage client expectations, issues and service performance.&Accountability&CSMIC&$\ast$ \\ \hline 
%   Ownership&The level of rights a client has over software licenses, intellectual property and data associated with a service.&Accountability&CSMIC&$\ast$ \\ \hline 
%   Provider business stability&The likelihood that the service provider will continue to exist throughout the contracted term.&Accountability&CSMIC&$\ast$ \\ \hline 
%   Provider Certifications&The service provider maintains current certifications for standards relevant to their clients' requirements.&Accountability&CSMIC&$\ast$ \\ \hline 
%   Provider Contract/SLA Verification&The service provider makes available to clients SLAs adequate to manage the service and mitigate risks of service failure.&Accountability&CSMIC&$\ast$ \\ \hline 
%   Provider Ethicality&Ethicality refers to the manner in which the service provider conducts business; it includes business practices and ethics outside the scope of regulatory compliance. Ethicality includes fair practices with suppliers, customers, and employees.&Accountability&CSMIC&$\ast$ \\ \hline 
%   Provider Personnel Requirements&The extent to which service provider personnel have the skills, experience, education, and certifications required to effectively deliver a service.&Accountability&CSMIC&$\ast$ \\ \hline 
%   Provider Supply Chain&The service provider ensures that any SLAs that must be supported by its suppliers are supported.&Accountability&CSMIC&$\ast$ \\ \hline 
%   Security Capabilities&The capabilities of service providers to ensure application, data, and infrastructure securiy based on the security requirements of the client.&Accountability&CSMIC&$\ast$ \\ \hline 
%   Sustainability&The impact on the economy, society and the environment of the service provider.&Accountability&SMIC&$\ast$ \\ \hline 
%   Adaptability&The ability of the service provider to adjust to changes in client requirements.&Agility&CSMIC&$\ast$ \\ \hline 
%   Capacity&The maximum amount of a service that a service provider can deliver while meeting agreed SLAs.&Agility&CSMIC&$\ast$ \\ \hline 
%   Elasticity&The ability of a service to adjust its resource consumption to meet demand.&Agility&CSMIC&$\ast$ \\ \hline 
%   Extensibility&The ability to add new features or services to existing services.&Agility&CSMIC&$\ast$ \\ \hline 
%   Flexibility&The ability to add or remove predefined features from a service.&Agility&CSMIC&$\ast$ \\ \hline 
%   Portability&The ability of a client to easily move a service from one service provider to another with minimal disruption.&Agility&CSMIC&$\ast$ \\ \hline 
%   Scalability&The ability of a service provider to increase or decrease the amount of service available to meet client requirements.&Agility&CSMIC&$\ast$ \\ \hline 
% \hline
% \end{tabular}
% \caption{Key Performance Indicators I.}\label{table:kpi-1}
%   \end{center}
% \end{sidewaystable} 
% 
% \begin{sidewaystable}[!ht]
% \renewcommand{\arraystretch}{1.3}
% \tiny
% \begin{center}
% \begin{tabular}[c]{|p{2.5cm}|p{9cm}|p{1.5cm}|p{1.5cm}|p{1.5cm}|} 
% \hline
%   \textbf{KPI} &  \textbf{Definition}  &  \textbf{Broader} & \textbf{Defined by} & \textbf{Scope} \\\hline
%  Availability&The amount of time that a client can make use of a service.&Assurance&CSMIC&$\ast$ \\ \hline 
%  Maintainability&Maintainability refers to the ability for the service provider to make modifications to the service to keep the service in a condition of good repair.&Assurance&CSMIC&$\ast$ \\ \hline 
%  Recoverability&Recoverability is the degree to which a service is able to quickly resume a normal state of operation after an unplanned disruption.&Assurance&CSMIC&$\ast$ \\ \hline 
%  Reliability&Reliability reflects measures of how a service operates without failure under given conditions during a given time period.&Assurance&CSMIC&$\ast$ \\ \hline 
%  Resiliency&The ability of a service to continue to operate properly in the event of a failure in one or more of its components.&Assurance&CSMIC&$\ast$ \\ \hline 
%  Stability&The degree to which the service is resistant to change, deterioration, or displacement.&Assurance&CSMIC&$\ast$ \\ \hline 
%  Serviceability&The ease and efficiency of performing maintenance and correcting problems with the service.&Assurance&CSMIC&$\ast$ \\ \hline 
%  Acquisition&Any client costs to acquire the rights and ability to use a service and to move from an existing service to the new one.&Financial&CSMIC&$\ast$ \\ \hline 
%  On-going Cost&The client cost to operate a service. This includes both recurring flat costs (e.g. monthly access fees) and usage-based costs.&Financial&CSMIC&$\ast$ \\ \hline 
%  Profit&Arrangement between client and provider(s) under which costs or profits of a service are shared by the involved parties, according to an agreed upon formula.&Financial&CSMIC&$\ast$ \\ \hline 
%  Accuracy&The extent to which a service adheres to its requirements.&Performance&CSMIC&$\ast$ \\ \hline 
%  Functionality&The specific features provided by a service.&Performance&CSMIC&$\ast$ \\ \hline 
%  Suitability&How closely do the capabilities of the service match the needs of the client.&Performance&CSMIC&$\ast$ \\ \hline 
%  Suitability&How closely do the capabilities of the service match the needs of the client.&Usability&CSMIC&$\ast$ \\ \hline 
%  Suitability&The ability of a service to easily interact with other services (from the same service provider and from other service providers).&Performance&CSMIC&$\ast$ \\ \hline 
%  Service Response Time&An indicator of the time between when a service is requested and when the response is available.&Performance&CSMIC&$\ast$ \\ \hline 
%  Access Control&Policies and processes in use by the service provider to ensure that only the provider and client personnel with appropriate status/reasons to make use of or modify data/work products may do so.&Security and Privacy&CSMIC&$\ast$ \\ \hline 
%  Geographical Constraint&The client's constraints on service location based on geography.&Security and Privacy&CSMIC&$\ast$ \\ \hline 
%  Political Constraint&The client's constraints on service location based on politics.&Security and Privacy&CSMIC&$\ast$ \\ \hline 
%  Data Integrity&Keeping the data that is created, used, and stored in its correct form so that clients may be confident that it is accurate and valid.&Security and Privacy&CSMIC&$\ast$ \\ \hline 
%  Data Privacy&Client restrictions on access and use of client data are enforced by the service provider. Any failures of these protections are promptly detected and reported to the client.&Security and Privacy&CSMIC&$\ast$ \\ \hline 
%  Data Loss&Client restrictions on access and use of client data are enforced by the service provider. Any failures of these protections are promptly detected and reported to the client.&Security and Privacy&CSMIC&$\ast$ \\ \hline 
%  Physical \& Environmental Security&Policies and processes in use by the service provider to protect the provider facilities from unauthorized access, damage or interference.&Security and Privacy&CSMIC&$\ast$ \\ \hline 
%  Proactive Threat \&  Vulnerability Management&Mechanisms in place to ensure that the service is protected against know recurring threats as well as new evolving vulnerabilities.&Security and Privacy&CSMIC&$\ast$ \\ \hline 
%  Retention/Disposition&The service provider’s data retention and disposition processes meet the clients' requirements.&Security and Privacy&CSMIC&$\ast$ \\ \hline 
%  Accessibility&The degree to which a service is operable by users with disabilities.&Usability&CSMIC&$\ast$ \\ \hline 
%  Client Personnel Requirements&The minimum number of personnel satisfying roles, skills, experience, education, and certification required of the client to effectively utilize a service.&Usability&CSMIC&$\ast$ \\ \hline 
%  Client Personnel Requirements&Installability characterizes the time and effort required to get a service ready for delivery (where applicable).&Usability&CSMIC&$\ast$ \\ \hline 
%  Learnability&The effort required of users to learn to use the service.&Usability&CSMIC&$\ast$ \\ \hline 
%  Operability&The ability of a service to be easily operated by users.&Usability&CSMIC&$\ast$ \\ \hline 
%  Transparency&The extent to which users are able to determine when changes in a feature or component of the service occur and whether these changes impact usability.&Usability&CSMIC&$\ast$ \\ \hline 
%  Understandability&The ease with which users can understand the capabilities and operation of the service.&Usability&CSMIC&$\ast$ \\ \hline 
% 
% \hline
% \end{tabular}
% \caption{Key Performance Indicators II.}\label{table:kpi-2}
%   \end{center}
% \end{sidewaystable} 
% 
