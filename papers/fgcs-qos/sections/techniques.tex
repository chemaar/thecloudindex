The increasing number of applications and services on the cloud is creating an 
across computer platform to support new methods in different disciplines such as 
life sciences or engineering disciplines. The stream model for data intensive 
monitoring and analysis can potentially benefit these applications but the 
processing of data in real time must ensure that data is gathered continuously 
(24/7), large volumes of data are properly addressed taking into account that 
data sources are distributed and is often not feasible to store all data for 
processing at a later time, thereby, requiring real-time analysis. 

On the other hand, a sensor can be defined as a data source which can produces a 
sequence of data items over time~\cite{deri2010}. This definition leads us to a new 
environment in which any device can continuously produce data and we can take 
advantage of this information to deliver advanced services on monitoring 
distinct domains such as traffic, logistics or supply chain management, etc. 
Gartner predicts that ``By 2015, wirelessly networked sensors in everything we 
own will form a new Web. But it will only be of value if the \textit{terabyte torrent} 
of data it generates can be collected, analyzed and interpreted'' and this 
implies that new techniques and models should be developed to deal with this 
vast amount of data making the integration and interoperability among 
applications possible.

In the particular context of Cloud Computing each resource, application or 
service can be seen as a data sensor producing data and information about their 
current status that can be processed in an automatic way with the objective of 
enabling self-management cloud systems and applications. Nevertheless some 
problems arise to manage a stream of data: storage, semantics, inference and 
querying, among others. As authors present in~\cite{deri2010} traditional data stream 
management systems (DSMS) assumes stream data has a bigger impact in query 
performance than static metadata. In the case of Linked Data this situation is 
more dramatic due to the existence of different vocabularies and the possibility 
of having new RDF triples in execution time that are not modeled by static 
metadata. Existing triple stores do not manage this situation and assume RDF 
data is static. In a cloud system environment this assumption is no longer hold 
and a method to query and make decisions in real-time processing stream data 
must be provided in order to deliver a reactive knowledge-based system. More 
specifically in the QoS management context, the need of dynamic adaptation of 
resources is dramatically growing up due to the necessity of adjusting costs and 
optimizing the reservation of cloud resources. Thus the QoS management can be 
seen as a motivating scenario to apply stream reasoning over data coming from 
cloud resources.

First efforts to address the data stream challenges were made in the IEEE 1451 
definition, the Radiation Detection Standards (ANSI N42), the OGC Sensor Web 
Enablement (SWE)~\cite{sensorweb-wb} or the Extended Environments Markup Language~\cite{eeml} (EEML). 
They followed a domain-specific approach that cannot be easily applied in other 
contexts. A standard with a broader application was the Sensor Web Enablement 
standard which was implemented by the 52º North project~\cite{52project}, the NASA/JPL~\cite{nasajpl}
Sensor Webs Project~\cite{Sheth:2008:SSW:1444383.1444435} and the European Space Agency among others. 
Efforts with regards to Linked Data such as ``SensorBase'' enables the publication 
of sensor data via HTTP POST as well as Pachube that offers a RESTfull interface 
for querying stream real-time sensor data in different formats such as JSON or EEML. 
Nevertheless the main drawback of these approaches is the lack of semantic descriptions to describe data streams.


In~\cite{Sheth:2008:SSW:1444383.1444435} authors proposed the use of RDFa to annotate ontological concepts and 
properties to SWE by using XLink. Thus the sensor information can support 
semantic functionalities and new services can be implemented on the top of RDF. 
Furthermore Whitehouse et al. launched the SemanticStreams project~\cite{Whitehouse:2006:SSF:2180141.2180148} based on 
Prolog rules to allow user queries on the semantics of sensor and in~\cite{Bouillet:2007:SMU:1769087.176909} 
authors propose to use OWL to represent data stream and provide a semantic 
context in which the composition of applications could be easily done. However 
they assume~\cite{deri2010} that semantic descriptions are already available and have 
enough quality. Finally the W3C Semantic Sensor Network Incubator Group has 
proposed the SSN Ontology~\cite{Compton:2012:OPS:2400766.2401456} to answer the need of a domain-independent and 
end-to-end model for sensing application by merging sensor focused (e.g. SensorML), observation-focused (e.g. observation and measurement) and system 
focused views. It also covers subdomains which are sensor-specific such as the 
sensing principles and capabilities and can be used to define how a sensor will 
perform in a particular context. Nevertheless further steps imply their real 
adoption in a Linked Sensor Data context to bridge the new era of Internet of 
Things and Internet of Services fostering their potential adoption in the OGC 
community.

Following the review of approaches to mix semantic technologies and stream data processing is 
presented with special focus on storage, stream processing, analytics and some applications:
\begin{itemize}
 \item In the case of storage, RDF triple stores such as Sesame, Apache Jena 
(Joseki), RISC-3X, YARS2, Oracle Semantic, OWLim, OpenLink Virtuoso have gained 
importance in recent times due to the deployment of the Linked Open Data 
initiative~\cite{Auer:2007:DNW:1785162.1785216}. Most of them are based on a SQL-backend and offer distinct 
services such as basic reasoning processes and a query service via a SPARQL 
endpoint. The main focus of these repositories is the scalability in terms of 
the number of triples they can manage. There is mainly adequate when data do not 
change frequently and a better performance can be reached applying well-known 
techniques of indexing. Moreover the current initiative of NoSQL databases~\cite{NoSQLdistilled} 
brings the opportunity of using new representation and retrieving models, see 
Table~\cite{nosql-db},  to deal with the vast amount of data in contexts such as Social Media 
or in our case of study data coming from dozen of applications.

Furthermore Apache Hadoop~\cite{perera2013mapreduce,owens2013mapreduce} and projects built on top of this Map/Reduce 
framework~\cite{Dean:2008:MSD:1327452.1327492} such as Apache Pig, Hive, Mahout, Cassandra or Zookeeper enable a 
distributed processing environment of large data sets across clusters of 
computers using simple programming models. The storage and processing models of 
large datasets is not a mere question and it must be carefully planned to avoid 
scalability problems. In this context there are also some projects providing 
data flows such as Storm~\cite{BigDataManing} (a distributed and fault-tolerant teal-time 
computation environment), most of these efforts are boosted by the main social 
media sites such as Facebook, Twitter or Linkedin. 

\item There is a growing set of both commercial and open tools that 
claims to provide large scale and distributed data stream processing, inspired in Apache Drill~\cite{hausenblas2013apache}. 
All of them use their own internal data models, cluster processing, a formal query language and different kinds of 
predictive analytics. In this context Impala~\cite{impala-project}, SPARK~\cite{zaharia2012discretized}, Sparrow~\cite{ousterhout2013sparrow},
Shark~\cite{DBLP:conf/sigmod/XinRZFSS13}, S4~\cite{neumeyer2010s4}, Druid~\cite{yangdruid} or MapR~\cite{mapr-project} technologies to name a 
few deliver such predictive services on big data through off-line and real-time queries. Usually they 
also take advantage of: 1) programming languages with functional and parallel capabilities such as Scala, Clojure, Erlang or Python; 
2) NoSQL stores and 3) MapReduce-based frameworks such as Apache Hadoop. Finally an advanced dashboard to manage jobs is 
commonly provided as a graphical interface for controlling the distributed execution and the results of predictive 
analysis functions.


\item Data stream management systems (DSMS) are other possibility to manage data streams GraphgCQ, Amazon/Cougar, Aurora, Gigascope, 
Hancock, Niagara, OpenCQ, Stream, Stream Mill, Tapestry, Tribeca, Streambase, Coral8, Apama or Truviso among others were implemented to overcome the 
limitation of traditional DBMS when continuous data updates appear. They are based on different data models and their implicit semantics is not based on static 
restrictions such as the ACID principles. The main objective lies in getting a better performance when huge amounts of data should be managed. According to~\cite{deri2010} and 
taking into account there is no benchmark available about their support of Linked Data streams these management systems are not suitable for processing RDF streams. 


\item StreamingSPARQL~\cite{Bolles:2008:SSE:1789394.1789438} and 
C-SPARQL~\cite{Barbieri:2010:EEC:1739041.1739095} purpose extensions to the 
existing SPARQL language to register queries in a time frame or sliding windows. 
StreamSPARQL presents a rather simple query evaluation model without taking into 
account performance issues whilst C-SPARQL enables an execution framework built 
of top of existing stream data management and triple storage systems. In that 
sense C-SPARQL is supposed to be more stable than StreamSPARQL and it also 
provides a service for splitting the continuous queries into two groups: static 
and dynamic which are merged through orchestration service binding the required 
facts when new RDF triples arise. A complete summary and tutorial on semantic 
stream reasoning is provided in~\cite{stream-reasoning-linked-data}. Authors also outline all the features that 
stream reasoning should address such as the use of continuous semantics instead 
of ``static'' semantics. In~\cite{deri2010} authors present the Continuous Query Evaluation 
over Linked Streams (CQELS) project to unify data coming from both the Linked 
Open Data cloud and sensor stream data. They focus on continuous queries that 
is, queries that are registered in the system, and executed every time new data 
matches their criteria considering the queries themselves as input parameters. 
Finally a complete study of stream processing in the cloud 
is also depicted in~\cite{Backman:2012:MPS:2169090.2169091}.

\item On the other hand, the CEP community is concerned with timely detection of 
compound events in streams of simple events. Nevertheless event processing 
cannot combine streams with background knowledge and cannot perform reasoning 
tasks. Although the use of event processing engines is widely accepted and 
technology such as Drools Fusion is well-known and used in production 
environments the use of semantic web technologies can perfectly manage 
background knowledge, mix different data streams and execute reasoning 
processes. That is why the conjunction of event processing for dealing with a 
vast amount of data and semantics to manage several streams and background 
knowledge can improve the exploitation of diverse data streams. In this context 
EP-SPARQL~\cite{Anicic:2011:EUL:1963405.1963495} is an extension of the existing SPARQL language to support complex 
events and stream reasoning. The main challenge of this work is to take 
advantage of real-time data, and recognizes important situations of interest in 
a timely fashion. Authors conclude that EP-SPARQL specifies complex events by 
temporarily situating real-time streaming data, and uses background ontologies 
to enable stream reasoning.

\item The problem of querying large datasets and performing semantic reasoning 
is being addressed in WebPIE~\cite{DBLP:journals/ws/UrbaniKMHB12} and QueryPIE~\cite{Urbani:2011:QBR:2063016.2063063} which are prototypes 
implemented in the context of the LarKC project. These projects show how to use 
a Map/Reduce framework for building an inference engine. WebPIE is focused on 
the calculation of the transitive closure of OWL semantics, more specifically 
RDFS and OWL-Horst~\cite{terHorst:2005:CDC:1741305.1741318}. This inference engine proposes a scalable technique to 
parallelize OWL Horst forward inference over $100$ billion of RDF triples. The 
main drawback of this approach lies in the necessity of executing the whole 
reasoning process when new facts arise that is why it cannot be considered 
adequate for stream reasoning. On the other hand and with the objective of 
processing dynamic data QueryPIE offers a backward inference engine to execute 
queries over large datasets performing a reasoning process before retrieving 
results. This approach is similar to the traditional backtracking technique used 
in Prolog engines. In this context of reasoning over large datasets, in~\cite{Hogan:2008:SAR:1484143.1484150,DBLP:journals/ijswis/HoganHP09,Umbrich:2012:IRL:2415113.2415129,Umbrich:2012:FUW:2413941.2413963} 
a system to compute the closure of RDF graphs is also presented but it 
only supports a fragment of OWL Horst to enable efficient materialization and 
(live) query processing on the Linked Data realm. In~\cite{Weaver:2009:PMF:1693684.1693728} a straightforward 
parallel RDFS reasoning on a cluster is also presented.

\item As we have seen in the previous works the implementation of SPARQL 
extensions to be executed in a distributed environment is currently a good 
approach to deal with a mass of data and provide a scalable environment for the 
Semantic Web. In this sense, authors in~\cite{DBLP:journals/pvldb/HuangAR11} make an implementation of a SPARQL 
distributed engine to split data and queries into different nodes to get 
local-optimizations. In~\cite{Schatzle:2011:PMS:1999299.1999303} authors propose a translation of SPARQL queries 
into Pig Latin for the scalable processing of complex queries on very large RDF 
datasets. They introduce PigSPARQL, a system which processes the SPARQL queries 
as Map/Reduce jobs, and make an evaluation using the benchmark SP2Bench (a 
specific SPARQL performance benchmark) getting good results. Following a similar 
approach HadoopSPARQL~\cite{liuhadoopsparql} is a tool that allows user to submit multiple queries 
at the same time. These queries are handled by an algorithm that is in charge of 
creating and distributing sub queries using Map/Reduce jobs. In~\cite{FarhanHusain:2009:SRL:1695659.1695734} authors 
present a work to store RDF in HDFS (Hadoop File System) and an algorithm to ask 
SPARQL queries. H2RDF~\cite{Papailiou:2012:HAQ:2187980.2188058}, Jena-HBase~\cite{DBLP:conf/semweb/KhadilkarKTC12} and the proposal in~\cite{maindonald2007data} 
are similar works trying to manage large RDF datasets and execute distributed SPARQL queries. 

\item The use of the R statistics package is widely accepted in the Big 
Data community to make decisions, predictions or analysis and perform statistical methods on large datasets. In 
the context of SPARQL, there is a specific package, \texttt{r-sparql}, to work with R that enables the 
connection to existing RDF endpoints from the R processor. The analysis and exploitation of large datasets is not easy to solve and data 
mining frameworks as Weka or Apache Mahout offer scalable methods and algorithms 
to detect patterns, etc. in data. The main drawback of these tools lies in the 
lack of integration with semantic technologies. Another problem arises when a 
large RDF dataset must be processed, mainly due to the URIs. To tackle this 
problem there are compacted formats such RDF Binary~\cite{DBLP:journals/ws/FernandezMGPA13} for publishing and exchanging RDF triples.


\item Regarding streaming applications, in~\cite{BLP:conf/cloudcom/VijayakumarZA10} two applications are described 
and used as benchmarks in the data mining domain. The first one, CluStream, is a 
cluster evolving data streams~\cite{Aggarwal:2003:FCE:1315451.1315460} that groups similar objects or data points 
from a given set into clusters. They address the problem of clustering data when 
data is arriving from continuous streams and changing over the time. Firstly, 
mini-clusters are created to compute statistical information. After that a 
modified version of the K-means algorithm is applied to create the final output 
of clusters. The second algorithm studies the frequency of occurring items in a 
distributed data stream, Approx-Freq-Counts~\cite{Schneider:2009:ESD:1586640.1587424}. Two parameters are defined: 
support and error, each node observes the frequencies of item sets in each 
stream and periodically sends this information to a parent node. These two 
approaches are considered to solve the problem of resource provisioning for data 
stream applications in virtualized or cloud environments finding dynamic 
patterns when data arrive.  The resource provisioning algorithm correctly 
converges to the optimal CPU allocation based on the data arrival rate and 
computational needs. The algorithm identifies over-flow and under-flow 
conditions and converges to the same level, irrespective of the initial 
allocation. The main advantage of this approach is that the system can 
automatically tune itself based on resource needs. 

\item Finally in~\cite{Ishii:2011:ESC:2055437.2055625} authors present the ElasticStream system that dynamically 
allocates computational resources on the cloud in an elastic manner for a data 
stream processing application. To minimize the charges for using the Cloud 
environment while satisfying the SLA, they formulate a linear programming 
problem to optimize the costs as a trade-off between the application’s latency 
and charges. A system is also implemented to assign or remove computational 
resources dynamically on the top of the data stream middleware. Authors also 
conclude that their approach could save up to 80\% of the costs while maintaining 
the application’s latency in comparison to a naïve approach. Nevertheless they 
remain as future work the prediction of data rate and the inclusion of new 
features in the platform to a better stream data management. This approach does 
not use any semantic feature to address the needs of managing a cloud 
environment and it is just based on an optimization algorithm to select 
adequately the allocation of resources in the cloud.

\end{itemize}

\subsection{Summary and Evaluation}
This section presents a summary of the most relevant aforementioned techniques with the aim
of establishing a factsheet and a tool for making decisions when a technique 
has to be selected for some use. We have created a set of features, see Table~\ref{features-techniques}, 
to list those references that present some evidence for validating its application 
and contribution to datastream and big data processing. This list is not exclusive 
but remarks main characteristics to take into account when a monitoring tool/technique 
is required. Each feature is evaluated following the next approaches:
\begin{itemize}
 \item Open ended questions using a word/sentence associated to the feature, for instance ``Apache Hadoop'' or ``SQL''.
 \item Multiple choice using the Likert scale~\cite{albaum1997likert} value: 1-Strongly disagree, 2-Disagree, 3-Neither agree nor disagree, 4-Agree and 5-Strongly agree.
 \item Closed ended questions with a Yes (Y)/No (N) value.
 \item Finally the symbol ``-'' is used to represent those unkown/missing/not applicable features.
\end{itemize}


% \begin{table}[!ht]
% \renewcommand{\arraystretch}{1.3}
% \tiny
% \begin{center}
% \begin{tabular}[c]{|p{1cm}|p{2.5cm}|p{5cm}|p{3cm}|} 
% \hline
%   \textbf{Feature}& \textbf{Label} &  \textbf{Definition}  &  \textbf{Type} \\\hline
%   $f_1$&Data model & This feature indicates how data is represented & Word/sentence associated \\ \hline
%   $f_2$&Programming Language & The programming language used to implement this tool/technique & Word/sentence associated \\ \hline
%   $f_3$&Reasoning & The tool enables some kind of reasoning process & Yes/No and Word/sentence associated  \\ \hline
%   $f_4$&Input Format & The input formats that the technique can process & Word/sentence associated  \\ \hline
%   $f_5$&Output Format & The output format that the technique can generate & Word/sentence associated  \\ \hline
%   $f_6$&Data integration & The tool enables the consumption of different datasources &  Yes/No \\ \hline
%   $f_7$&Architecture & The type of architecture for data processing & Word/sentence associated\\ \hline
%   $f_8$&Programming Model & The type of programming model for data processing  & Word/sentence associated\\ \hline
%   $f_9$&API & The tool provides and API to manage all processes & Yes/No  \\ \hline
%   $f_{10}$&Accesibility & The tool can be easily used in different context & Likert scale  \\ \hline
%   $f_{11}$&Adaptability & The tool can be easily configured to meet new requiremens & Likert scale  \\ \hline
%   $f_{12}$&Auditability & The tool provides a mechanism to know how it is working & Likert scale  \\ \hline
%   $f_{13}$&Elasticity & The tool can be adjusted its resource consumption to meet demand & Likert scale  \\ \hline
%   $f_{14}$&Extensibility & The tool can be easily extended & Likert scale  \\ \hline
%   $f_{15}$&Flexibility & The tool can be configured on-demand adding/removing features & Likert scale  \\ \hline  
%   $f_{16}$&Interoperability & The tool can be integrated with third-parties  & Likert scale  \\ \hline
%   $f_{17}$&Portability & The tool can be easily move to different architectures & Likert scale  \\ \hline
%   $f_{18}$&Usability & The tool can be easily configured  & Likert scale  \\ \hline
%   $f_{19}$&Security & The tool can be used under a protocol for secure communication & Likert scale  \\ \hline
%   $f_{20}$&Standards & The tool is based in the use of standars (compliance) & Likert scale\ \\ \hline
%   $f_{21}$&Off-line processing & The technique enables off-line/batch processing & Yes/No  \\ \hline
%   $f_{22}$&Real-time processing & The technique enables real-time processing & Yes/No  \\ \hline
%   $f_{23}$&Storage & The database to store data &  Word/sentence associated \\ \hline
%   $f_{24}$&Query & The formal query language to accesss to results & Word/sentence associated\\ \hline
%   $f_{25}$&Dashboard & The technique provides a graphical tool to manage processes, export data, etc. & Likert scale\ \\ \hline
%   $f_{26}$&Licensing and pricing & The type of license & Word/sentence associated \\ \hline
%   $f_{27}$&Maturity & The level of maturity or development of the tool & Likert scale\\ \hline
%   $f_{28}$&Presence & The level of presence in existing products & Likert scale\\ \hline
%   $f_{29}$&Update and evolution & The tool is frequently updated & Likert scale\\ \hline
%   $f_{30}$&Certification & The tool provides courses and other methods to learn its use & Likert scale\\ \hline
%   $f_{31}$&Financial support & The tool is supported for investors/venture capital, etc. & Likert scale\\ \hline
% \hline
% \end{tabular}
% \caption{Features for selecting a technique of datastream processing.}\label{features-techniques}
%   \end{center}
% \end{table} 

\begin{table}[!ht]
\renewcommand{\arraystretch}{1.3}
\tiny
\begin{center}
\begin{tabular}[c]{|p{2.5cm}|p{5cm}|p{3cm}|} 
\hline
  \textbf{Feature} &  \textbf{Definition}  &  \textbf{Type} \\\hline
  Data model & This feature indicates how data is represented & Word/sentence associated \\ \hline
  Programming Language & The programming language used to implement this tool/technique & Word/sentence associated \\ \hline
  Reasoning & The tool enables some kind of reasoning process & Yes/No and Word/sentence associated  \\ \hline
  Formats & The input formats that the technique can process & Word/sentence associated  \\ \hline
  Data integration & The tool enables the consumption of different datasources &  Yes/No \\ \hline
  Architecture & The type of architecture for data processing & Word/sentence associated\\ \hline
  Programming Model & The type of programming model for data processing  & Word/sentence associated\\ \hline
  API & The tool provides and API to manage all processes & Yes/No  \\ \hline
  Accesibility & The tool can be easily used in different context & Likert scale  \\ \hline
  Adaptability & The tool can be easily configured to meet new requiremens & Likert scale  \\ \hline
  Auditability & The tool provides a mechanism to know how it is working & Likert scale  \\ \hline
  Elasticity & The tool can be adjusted its resource consumption to meet demand & Likert scale  \\ \hline
  Extensibility & The tool can be easily extended & Likert scale  \\ \hline
  Flexibility & The tool can be configured on-demand adding/removing features & Likert scale  \\ \hline  
  Interoperability & The tool can be integrated with third-parties  & Likert scale  \\ \hline
  Portability & The tool can be easily move to different architectures & Likert scale  \\ \hline
  Usability & The tool can be easily configured  & Likert scale  \\ \hline
  Security & The tool can be used under a protocol for secure communication & Likert scale  \\ \hline
  Standards & The tool is based in the use of standars (compliance) & Likert scale\ \\ \hline
  Off-line processing & The technique enables off-line/batch processing & Yes/No  \\ \hline
  Real-time processing & The technique enables real-time processing & Yes/No  \\ \hline
  Storage & The database to store data &  Word/sentence associated \\ \hline
  Query & The formal query language to accesss to results & Yes/No\\ \hline
  Dashboard & The technique provides a graphical tool to manage processes, export data, etc. & Likert scale\ \\ \hline
  Licensing and pricing & The type of license & Word/sentence associated \\ \hline
  Maturity & The tool has reached a good level of maturity or development & Likert scale\\ \hline
  Presence & The tool has reached a good level of presence in existing products & Likert scale\\ \hline
  Update and evolution & The tool is frequently updated & Likert scale\\ \hline
  Certification & The tool provides courses and other methods to learn its use & Likert scale\\ \hline
  Financial support & The tool is supported for investors/venture capital, etc. & Likert scale\\ \hline
\hline
\end{tabular}
\caption{Features for selecting a technique of datastream processing.}\label{features-techniques}
  \end{center}
\end{table} 


\begin{sidewaystable}[!ht]
\renewcommand{\arraystretch}{1.3}
\tiny
\begin{center}
%\begin{tabular}[c]{|p{0.5cm}|p{0.6cm}|p{0.6cm}|p{0.6cm}|p{0.6cm}|p{0.8cm}|p{0.6cm}|p{0.6cm}|p{0.7cm}|p{0.5cm}|p{0.6cm}|p{0.6cm}|p{0.6cm}|p{0.6cm}|p{0.6cm}|p{0.6cm}|p{0.6cm}|p{0.6cm}|p{0.6cm}|p{0.6cm}|p{0.6cm}|p{0.8cm}|p{0.8cm}|}
\begin{tabular}[c]{|p{1.7cm}|p{1.5cm}|p{1.5cm}|p{1.5cm}|p{1.5cm}|p{1.5cm}|p{1.5cm}|p{1.5cm}|p{1.5cm}|p{1.5cm}|p{1.5cm}|p{1.5cm}|} 
\hline
 % \textbf{Tool/Feature} & \textbf{Data model} & \textbf{Programming Language} & \textbf{Reasoning} & \textbf{Input Format} & \textbf{Output Format} & \textbf{Data integration} & \textbf{Architecture} & \textbf{Programming Model} & \textbf{API} & \textbf{Accesibility} & \textbf{Adaptability} & \textbf{Auditability} & \textbf{Elasticity} & \textbf{Extensibility} & \textbf{Flexibility} & \textbf{Interoperability} & \textbf{Portability} & \textbf{Usability} & \textbf{Security} & \textbf{Standards} & \textbf{Off-line processing} & \textbf{Real-time processing} & \textbf{Storage} & \textbf{Query} & \textbf{Dashboard} & \textbf{Licensing and pricing} & \textbf{Maturity} & \textbf{Presence} & \textbf{Update and evolution} & \textbf{Certification} & \textbf{Financial support} \\\hline
 \textbf{Tool/ Feature} & IEEE 1451 &  EEML~\cite{eeml} & SWE~\cite{sensorweb-wb} & SSN~\cite{Compton:2012:OPS:2400766.2401456} & Semantic Streams~\cite{Whitehouse:2006:SSF:2180141.2180148} & Storm~\cite{BigDataManing} & Impala~\cite{impala-project} & SPARK~\cite{zaharia2012discretized} & Druid~\cite{yangdruid} & S4~\cite{neumeyer2010s4}  & MapR~\cite{mapr-project}\\ \hline
  Data model & TEDS~\footnote{Transducer electronic data sheet} & EEML & SensorML & OWL / RDF & -  & Topology & - & - & - & - & - \\ \hline  
  Programming Language & - &  Java & - & -& - & Java and JVM-based languages & Java & Scala, Java \& Python. & Java & Java &  Java\\ \hline  
  Reasoning & - & - &  - & Y & Y & N & N & N & N &  N & \\ \hline  
  Formats & 1451.x & EEML & SensorML & RDF syntax~\footnote{RDF/XML normative syntax, Turtle, N3, etc.} & - & - & Sequence Files (Snappy, Avro, Gzip, Bzip, LZO etc.) & Hadoop Input Formats & - & - &Hadoop Input Formats \\ \hline  
  Data integration & Y & Y & Y  & Y & Y & N & N & N & N & N & N \\ \hline  
  Architecture & S/D~\footnote{Sensor/Distributed (cluster)} & S/D  & S/D  & S/D  & S/D & D & D & D & D  & D  & D  \\ \hline  
  Programming Model & Service & Service & Service & - & -  & MapReduce~\footnote{It is based on MapReduce in the sense of batch processing.} & MapReduce & MapReduce & MapReduce &EP~\footnote{Event Processing}  &  MapReduce\\ \hline  
  API & 1451.x & Pachube  & SOS~\footnote{Sensor Observation Service} and SPS~\footnote{Sensor Planning Service}  & - & Y & Y & Y & Y & Y  & Y & Y \\ \hline  
  Accesibility & 4 & 4 & 4 & 5 & 2 & 5 & 5 & 5 & 5 & 5 & 5 \\ \hline  
  Adaptability & 4 & 4& 5 & 5 & 4 & 5 & 5 & 5 & 4 & 5 & 5 \\ \hline  
  Auditability & 3 &3 & 3 & 3 & 3 & 5 &  5& 4 & 3 & 4 & 4 \\ \hline  
  Elasticity & 4 & 4& 4 & 5 & 4 & 5 & 5 & 5 &4  & 5 & 4 \\ \hline  
  Extensibility & 4 & 4& 4 & 5 & 4 & 5 & 5 & 5 & 4 & 5 & 5 \\ \hline  
  Flexibility &4  & 5& 5 & 5 & 4 & 5 & 5 & 5 & 4 & 5 & 5 \\ \hline  
  Interoperability & 4 &5& 5 & 5 & 4 & 4 & 4 & 4& 3 & 4 & 4\\ \hline  
  Portability & 4 &5 & 3 &5  & 4 & 5 & 5 & 5 & 3 & 5 & 5\\ \hline  
  Usability & 3 &4 & 4 & 1 & 4 & 4 &5  & 4 & 3 & 4 & 4\\ \hline  
  Security & 3 & 3& 3 & 3 & 3 & 4 & 5 & 3 & 3 & 3 & 4 \\ \hline  
  Standards & 5 & 4 & 5& 5 & 3 & 3 & 4 & 3 & 3 & 4 & 4 \\ \hline  
  Off-line processing & 3 &3  &3 & - & - & 2 & 4 & 4 & 4 & 3 & 5 \\ \hline  
  Real-time processing & 3 &4 & 4 & - & 4 & 5 & 54 & 5 & 5 & 5 & 4  \\ \hline  
  Storage & 3 &3 & 3 & 3 & 3 & 4 (DFS~\footnote{Distributed File System}) & HDFS & HDFS & -  & DFS & HDFS \\ \hline  
  Query & - & Y& Y & Y (SPARQL) & Y  & Y (Trident) & Hive SQL & Hive SQL (iif Shark)  & Y (SQL-based) &  - & Hive SQL / Pig\\ \hline  
  Dashboard & 3 & 4 & 4 & 1 & 4 & 2 & 5 & 3 & 3 & 3 & 4\\ \hline  
  Licensing and pricing & Open & Open & Fee on TGAR~\footnote{Total Gross Annual Revenue} & W3C & - & EPL v1.0 & Apache & Apache & GPL & Apache  & M3 Edition is free, M5 \& M7 commercial licenses\\ \hline  
  Maturity & 4 & 4&  4& 2 & 4 & 5  & 5  & 4 & 4 & 5 &4 \\ \hline  
  Presence & 4 & 4 & 5 & 2 & 2 & 4 & 5 & 4 & 5 & 5 &5 \\ \hline  
  Update and evolution & 4 & 4 & 5  & 4 & 2 & 5 & 5 & 5 & 5 & 4 &5\\ \hline  
  Certification & 4 &4 & 5 & 2 & 2 & 3 & 5 & 4 & 3 & 4 & 3\\ \hline  
  Financial support & 4 & 4&  5 & 2 & 3 & 4 & 5 & 4 (Apache Spark) & 4  & 4 (Apache S4) & 5 \\ \hline  
  
\hline
\end{tabular}
\caption{Selected Tools and Techniques (I).}\label{table:tools-1}
  \end{center}
\end{sidewaystable} 

\begin{sidewaystable}[!ht]
\renewcommand{\arraystretch}{1.3}
\tiny
\begin{center}
\begin{tabular}[c]{|p{1.7cm}|p{1.5cm}|p{1.5cm}|p{1.5cm}|p{1.5cm}|p{1.5cm}|p{1.5cm}|p{1.5cm}|p{1.5cm}|p{1.5cm}|p{1.5cm}|} 
\hline
 \textbf{Tool/ Feature}  &  Streaming SPARQL~\cite{Bolles:2008:SSE:1789394.1789438} & C-SPARQL~\cite{Barbieri:2010:EEC:1739041.1739095} & CQELS~\cite{deri2010} & EP-SPARQL~\cite{Anicic:2011:EUL:1963405.1963495} & WebPIE~\cite{DBLP:journals/ws/UrbaniKMHB12} & QueryPIE~\cite{Urbani:2011:QBR:2063016.2063063} & SAOR~\cite{DBLP:journals/ijswis/HoganHP09} & Pig SPARQL~\cite{Schatzle:2011:PMS:1999299.1999303}& Hadoop SPARQL~\cite{liuhadoopsparql} & H2RDF~\cite{Papailiou:2012:HAQ:2187980.2188058}   \\ \hline
  Data model & OWL / RDF & OWL / RDF   &  & OWL / RDF & - & - &  OWL / RDF & - & - & -\\ \hline  
  Programming Language & Java & Java  & Java & Java & Java &  Java & Java &  Java &  Java & Java \\ \hline  
  Reasoning & N & N  & N & Y & Y (RDFS / OWL-Horst backward) & Y (RDFS / OWL-Horst backward) & Y (OWL / RDFS) & N & N & N \\ \hline  
  Formats & RDF syntax & RDF syntax  &RDF syntax  & RDF syntax & Hadoop Input Formats & Hadoop Input Formats & - & Hadoop Input Formats & Hadoop Input Formats & RDF syntax\\ \hline  
  Data integration & Y & Y & Y & Y &  N & N & N & N & N & Y\\ \hline  
  Architecture & C~\footnote{Centralized} & C  &  D& C & D & D & D & D & D & D\\ \hline  
  Programming Model & MapReduce & Events  & Events & Events &  MapReduce &  MapReduce & - & MapReduce & MapReduce & MapReduce\\ \hline  
  API & N & Y  & Y & Y & N & N & N & N & N & N \\ \hline  
  Accesibility & 3 & 4  & 4  & 4 & 4 & 4 & 4 & 3 & 2 & 3\\ \hline  
  Adaptability & 4 & 3  & 4 & 4 & 3 & 4 & 4 & 3 & 3 & 3\\ \hline  
  Auditability & 2 & 2  & 3 & 3 & 3 & 3 & 3 & 2 & 2 & 3\\ \hline  
  Elasticity & 3 & 2  & 4 & 4 & 3 & 4 & 4 & 4 & 4 & 4\\ \hline  
  Extensibility & 3 & 2  & 4 & 4 & 4 & 4 & 3 & 3 & 3 &4 \\ \hline  
  Flexibility & 2 & 2  & 3  & 4 & 3 & 3 & 3 & 4 & 3 & 4\\ \hline  
  Interoperability & 4 & 3  & 4 &4  & 4 & 4 & 4 & 3 & 3 & 4\\ \hline  
  Portability & 4 & 3  & 4 & 4 & 4 & 4 & 4 & 4 & 3 &4 \\ \hline  
  Usability & 2 & 2  & 3 & 4 & 4 & 4 & 4 & 4 & 3 & 4\\ \hline  
  Security & 3 & 2  & 3 & 3 & 3 & 3 & 3 & 3 & 3 & 3\\ \hline  
  Standards & 4 & 4  & 4 & 4 & 4 & 4 & 4 & 3 & 3 & 4\\ \hline  
  Off-line processing & 4 & 2  & 2 & 2 & 4 & 4 & 4 & 4 & 4 & 4\\ \hline  
  Real-time processing & 2 & 4  & 4 & 4 & 2 & 2 & 2 & 2 & 2 & 3\\ \hline  
  Storage &  M~\footnote{Memory}&M  & - & M &  HDFS & HDFS & - & HDFS & HDFS &HDFS \\ \hline  
  Query & Y (SPARQL) & Y (SPARQL)  & Y (CQELS language) & Y (SPARQL) & Y (SPARQL) & Y (SPARQL) & - & Y (SPARQL / Pig) & Y (SPARQL) & Y (SPARQL)\\ \hline  
  Dashboard & N & N &  & N & N & N & N & N & N & N\\ \hline  
  Licensing and pricing & - & CC BY-NC-SA 3.0  & LGPLv3.0 & GPL (Etalis Engine) & Apache & Apache & - & -  & - & GPL \\ \hline  
  Maturity & 2 &  3 & 4 & 4 & 4 & 4 & 4  & 3 & 2 & 3\\ \hline  
  Presence & 2 &  3 & 3 & 4 & 2 & 2  & 3 & 2 & 2 & 2\\ \hline  
  Update and evolution & 2 & 2  & 4 & 4 & 2 & 2 & 3 & 3 & 2 & 3\\ \hline  
  Certification & 1 & 1  & 3 & 3 & 2 & 2 & 2 & 3 & 2 & 3\\ \hline  
  Financial support & 1 & 1  & 3 & 3 & 2 & 2 & 3 & 3 & - & 3\\ \hline  

\hline
\end{tabular}
\caption{Selected Tools and Techniques (II).}\label{table:tools-ii}
  \end{center}
\end{sidewaystable} 

