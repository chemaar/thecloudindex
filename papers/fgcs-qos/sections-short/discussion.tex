Despite the growing interest in Cloud Computing and the hype of this 
paradigm for the creation of new era of applications, services and data 
available through the Web, a real advanced cloud management environment is far 
from being fully developed. There are many open issues to be solved and 
technology to ease the transition from traditional developments and applications 
to a cloud-based environment is still under development. With regards to QoS, 
there are a lot of initiatives and efforts trying to model and manage functional 
and non-functional properties in an intelligent fashion. Nevertheless the lack 
of standards for unifying information and data is preventing the deployment of 
advanced techniques for QoS management. In the case of semantic technologies, 
works in different areas are emerging to solve interoperability and integration 
problems in distributed environments. More specifically, the creation of 
knowledge-based systems applying semantic-based techniques as stream reasoning 
and CEP are currently being developed to deal mainly with Big Data problems in the 
context of social media or e-Government. Following a list of questions/answers 
are provided to discuss the current status and future challenges in the 
topics covered in this paper:
\begin{itemize}

 \item Which dimensions and metrics should be taken into account to manage QoS in Cloud Systems?
 
There are a big variety of QoS dimensions to ensure in a cloud system. 
The methods to ensure reliability, security and trust should be modeled and 
discovered in automatic ways. In this specific case it is also required to take 
into account user feedback to evaluate the real quality and trust of a service. 
Moreover, depending on the cloud layer, specific QoS characteristics should be 
defined to collect the requirements of each particular case. Currently QoS 
approaches are mainly focused on web service discovery and selection but new 
ranking methods~\cite{DBLP:journals/fgcs/GargVB13} and reactive control systems taking into QoS features should be 
deployed to provide an intelligent cloud infrastructure.  

\item Which is the ``best'' approach to tackle the QoS management in Cloud Systems?

From a policy-making perspective the use of quantitative indexes is a widely accepted 
practice to compile in just one value a set of indicators. In this sense the on-going 
works presented in Section~\ref{qos-cloud-index} are a nice start, more specifically 
the SMI index seems to be a clear candidate to assess the quality of cloud services (XaaS). 
Nevertheless it is necessary to find a method to: 1) integrate different data sources; 
2) model the index structure and its computation and 3) provide proper documentation 
in a multilingual environment with different user profiles and intentions. In this sense, 
semantics can help to address these requirements through a common and shared data model 
with implicit support for multilingual documentation. In fact if a quantitative index 
is modeled using semantics it can be deployed and transformed to an existing monitoring 
tool since RDF, OWL and SPARQL semantics is clearly defined and enable transformations 
to other formal models or languages such as SQL.

\item How QoS can leverage semantics?

According to Section~\ref{qos-semantics} the big variety of ontologies, OWL models, etc. that 
have been designed in recent years imply a tangled set of options that should be 
unified to provide an unique view of what QoS should be and cover. Apart from that the 
use of semantics is not clear, in some cases reasoning processes are used to 
discover services but others just define an ontology as a proposal to provide a 
formal model without any real application. A clear semantic-based architecture 
should be defined containing: 1) the adequate definitions of 
functional and non-functional properties and 2) different perspectives (technical and business).

Furthermore semantics can help to increase the reliability in Cloud Computing providing the 
building blocks and models for an advanced, standardized and inter-operable QoS management. 
In the same way, Cloud Computing can help semantic technologies to be more scalable and flexible making use of the web as 
infrastructure to create large-scale data-intensive batch applications.


\item Are semantic technologies able to enable scalable predictive analytical processes?

With regards to Semantic Web and reasoning, there is a growing community 
trying to provide technology for supporting intelligent systems in the new Web 
of Data. As a consequence the necessity of dealing with Big Data problems and 
data coming from different sources is stimulating the creation of new approaches 
to reuse existing technology such as Apache Hadoop in the context of querying 
large datasets. Therefore the main application of the Semantic Web principles 
lies in the unification of data and the execution of reasoning processes to 
validate data and infer new facts. Nevertheless, the existing logic formalisms 
available in OWL such as DL, FOL, F-Logic, etc. do not seem to be a solution to 
tackle the challenge of modeling dynamic domains that is why some works 
regarding Continuous Semantics are emerging.

\item How Linked Data can help to a better QoS management experience?

Currently this initiative has been successfully applied to 
information retrieval systems or in the creation of rich user interfaces. 
Nevertheless, the expectations of linking different datasets to enrich 
information are growing as a manner for delivering more intelligent services. To 
apply Linked Data in the Cloud Computing environment we should ensure that any 
resource to be monitored has an URI, its data is coded into RDF according to a 
formal model, an API or endpoint is accessible for fetching data using pulling, 
pushing or triggering techniques and the methods for graph processing and 
reasoning are efficient and scalable under real-time constraints.

\item How can I select a monitoring tool? Should I design and implement a new one from the scratch?

Following the review in Section~\ref{data-stream} there is already a proven set of tools to analyze 
big data. Most of them are based on an internal model, a formal query language and a NoSQL-based storage 
to perform queries or analysis in real-time. Nevertheless an extra layer of semantics based on standards 
such as RDF or SPARQL is still an open issue. An effort to expose processed data as a SPARQL endpoint it is 
completely possible since a similar approach have been reached using SQL (e.g. SploutSQL). In that sense a good approach can 
be to re-use the effort of existing monitoring tools but adding a RDF layer, see Section~\ref{framework}, 
to leverage the two main advantages of semantics: integration and interoperability.

\item Which are the key-factors to select a monitoring tool?


Depending on the context and the requirements of the problem (real-time, predictive analytics, type of license, etc.) 
some different characteristics must be evaluated but, at least, we should ensure if the monitoring tool 
is able to provide the next services: 1) stream processing and incremental calculation of statistics; 
2) paralleling processing; 3) a dashboard for data exploration or integration with other 
existing visualization tools; 4) data import capabilities; 5) extensibility; 6) use of standards and 6) in general, 
a reporting tool to extract summaries. In fact, the selection of a monitoring tool does not differ from 
any other kind of software but the two first points must be carefully evaluated. Finally the selection 
must be aligned with a business/research strategy.


\item Which is the next big thing in QoS management, Semantics and Big Data?

In the case of QoS management, if we assume we are able to access different key performance indicators and perform 
some analysis then the scenario presented in Section~\ref{motivating}, Cloud Brokerage, can 
be efficiently address. Nevertheless the complete automation of a broker service is still 
an utopia since human-validation is required to make strategic decisions. Furthermore it is necessary 
to define a QoS API, maybe using the SMI indicators, to be able to implement a real Cloud Quality Management 
Platform. For instance, SalesForce offers~\footnote{\url{http://trust.salesforce.com}} 
some RSS feeds to check the status of their services, a similar approach should be followed for other providers to boost 
the user experience and trust and create a real cloud service market.

On the other hand semantic technologies and Linked Data are now focused on addressing some challenges in
data quality~\cite{bizer2007,Bizer2009QA,lodq,link-qa}, provenance~\cite{DBLP:conf/ipaw/HartigZ10}, trust~\cite{Carroll05namedgraphs}, 
large dataset processing~\cite{DBLP:conf/closer/HausenblasGHC12}, entity reconciliation~\cite{Maali_Cyganiak_2011}, 
searching~\cite{hoga-etal-2011-swse-JWS} or inference~\cite{DBLP:journals/ws/BonattiHPS11} to name a few.

Finally the use of Big Data techniques and tools is already in the market and an evolving community is generating 
more and more approaches to provide more faster and scalable analysis processes. The design of algorithms taking 
advantage of a distributed environment, the possibility of integrating diverse data, the use of standards and the creation 
of more enriched visualization tools are some open issues that must be addressed to improve and bring these tools 
to public at large boosting a new data-based economy.

\end{itemize}




%  \item Information about a service consists in both static and dynamic data, the 
% matchmaking of services according to these issues is a key-enabler for a real 
% QoS management but it is not fully addressed by existing works.



