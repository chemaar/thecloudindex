In this section a literature review of main ontology-based frameworks for QoS management is presented. After that 
an empirical evaluation of some selected features, see Table~\ref{features-qos-models}, is also outlined to finally present 
a summary, see Table~\ref{summary-features-qos-models}, of the most relevant approaches for semantic-based QoS management.
% These issues are considered to be key-enablers of the future semantic-based QoS management in cloud environments.
\subsection{Ontology-based frameworks for QoS management}
Ontology-based resource description is proposed to solve problems in~\cite{Pernas:2005:UOD:1068510.1069326,Armstrong17} with regards to the difficulty of 
resource information management, no standard definitions of resource requirements and the difficulty of guaranteeing compatibility of resource allocation. 
There are works that tries to produce a global ontology by merging ontologies of resource groups~\cite{Lopes:2006:PEM:1135771.1136110}. 
Authors in~\cite{Ejarque:2008:USR:1443230.1444322} propose the Semantically-Enhanced Resource Allocator (SERA), a scheduling system using customer 
requests with the ability of re-scheduling requests based on their priorities and considering advanced reservations.

% In~\cite{2009gdc..conf..221Y} authors propose a resource virtualization method using a virtual ontology (VOn) and an execution environment called OReSS 
% (Ontology-based Resource Selection Service) that is configured dynamically based on user requirements. The VOn is mapped to cloud resources 
% and a Map/Reduce technique is applied for the rapid and efficient merging of a number of ontologies. The execution environment is comprised of resources 
% described using the VOn and they are automatically populated applying the Ontology Merge engine. The main contribution of this work lies in 
% the resource management in cloud computing systems tackling the complex management of distributed heterogeneous resources causes the aforementioned problems. 
% In order to solve these problems a resource virtualization method is proposed using ontologies in a cloud environment. 

In~\cite{rule-2013-resource-provisioning} a Rule Based Resource Manager is proposed for a cloud hybrid environment with the objective of increasing the scalability of private 
clouds on-demand being cost-effective. This work also set up the execution time for public and private cloud in order to fulfill requests selected different services. 
The methodology is applied to the IaaS layer to access resources on demand enabling the scale up of private clouds with a cost-effective.

The SITIO platform~\cite{Garcia-Sanchez:2010:ASS:1852403.1852409} gathers the emerging concepts of SaaS, semantic technologies, Business Process 
Modeling and Cloud Computing to foster dramatic evolution of a new platform oriented towards interoperability and cost reduction. SITIO is defined as a 
platform for reliable, privacy-aware, secure and cost-efficient semantics-based Software-as-a-Service Creation, Integration and Management. The relevant component 
of the SITIO platform lies in the annotation of web services using old-fashioned semantic web services techniques. Authors have implemented a methodology 
to enable the semi-automatic annotation of web services in a three-step method: 1) collect information from the web service; 2) find mappings between 
domain ontologies and the web service and 3) web service annotation and expert user validation of the suggested annotations. Although the SITIO platform applies 
semantics to solve interoperability problems it is only focused on the description of web services functionality and capabilities.

A declarative system called CloudRecommender is presented 
in~\cite{DBLP:conf/gecon/ZhangRNMH12} through a unified and formalized domain 
model capable of describing infrastructure services such as Amazon, Microsoft 
Azure, GoGrid, etc. A prototype is also presented to show the main benefits of 
this approach: 1) a recommender with the capability of estimating costs across 
multiple providers, 2) aid in the selection of cloud services and 3) a 
user-friendly service interface based on widgets that maps user requirements 
based on form inputs to available infrastructure services. Nevertheless authors 
suggest some future work including the extension of the recommender to support 
the selection of more cloud service types such as PaaS services and the 
integration with other existing cloud services as well as the implementation of 
benchmarking methods for facilitating run-time selection based on dynamic QoS 
information such as throughput, latency, and utilization.

In~\cite{5682131} authors review three concepts developed in the framework of 
the FP7 4WARD with the objective of demonstrating their potential impact on QoS 
management: network virtualization, generic path semantic resource management 
and in-network management. These are novel concepts that are being targeted at 
handling QoS issues and are supposed to be relevant for enabling a new dynamic, 
flexible, adaptable and scalable cloud environment. Thus network virtualization 
decouples networks from infrastructure and allows infrastructure resources to be 
shared among multiple isolated networks overcoming the limitations of the 
traditional “one-size-fits-all” approaches. However, a two-layer QoS model is 
required, which raises new challenges, particularly in multi-domain scenarios. 
The Generic Path is a new end-to-end communication abstraction that aims at 
overcoming the inadequacies of the traditional layered network model. Resource 
management is accomplished by applying shared semantic concepts and formalizing 
the heterogeneous communication technology with ontologies. The QoS features of 
the Generic Path can be represented by an ontology from which QoS profile 
parameters, such as bandwidth, delay and error tolerance can be derived. 
Semantic resource management enables fair resource strategies by combining best 
effort and strict allocation policies. Finally, the component in-Network 
management allows the incorporation of QoS management capabilities in network 
elements, facilitating QoS configuration.

Authors in~\cite{DBLP:conf/soca/ChenL10} aims to provide a suitable service cater to discover consumer 
service requests including functional requirements and non-functional 
properties. They propose a service registry model named as SRC which is an 
extension of the keyword based service registry model. The SRC is deployed as a 
cloud application to provide a behavior-aware and QoS aware service discovery 
storing semantic descriptors of web services and the feedback of dynamic status 
of QoS in a GFS file system. This data is processed using a Map/Reduce 
mechanism. Basically it is a matchmaking service based on the WSDL descriptions 
taking advantage of using OWL for simple annotations of functional and 
non-functional properties. Thus the system guarantees that the records of 
inputs, outputs, predicates, and constraints of a service are stored in the same 
node of the cloud and the OWL classes in every node of the cloud. That is why, 
even if the tables are divided into chunks and each chunk is stored in an 
exclusive node, when they look for all the semantic information of a service, 
they do not need to do any inter-node query. The main drawback of this approach 
lies in the necessity of ensuring synced multiple copies of OWL definitions on 
all nodes.

In~\cite{cloudle} a search engine and an architecture for cloud systems (Cloudle) is 
outlined to semantically look up services according to a user profile. The 
Cloudle system works as follows: a user send a query through a web interface, 
cloud providers are registered in the system with a rating, a service discovery 
agent works by means of a query processor that takes the user profile and 
performs a similarity reasoning process against registered services trying to 
optimize price and time-slots. Two ontologies have been also designed in order to 
assist this similarity reasoning process and are used to improve the accuracy of 
results. The difference between these two ontologies lies in their structure, 
the first one only includes a concept hierarchy and the second one also includes 
individuals, semantic relationships, etc. that improves the performance of the 
similarity matchmaking process providing reasoning over data type and object 
properties and concepts. The main finding of this study is that an enriched 
ontology can improve the selection of cloud services. However all concepts, 
properties, etc. are defined by the authors from the scratch without any reuse 
of existing standard concepts.

In~\cite{6206823} authors introduce the system Inter-cloud Resource Provisioning System 
(IRPS) to accomplish the requirements of a customer to the maximum providing 
additional resources to the cloud system participating in a federated 
environment. This system schedules some tasks to allocate resources by using 
semantics and an inference engine; more specifically they use Sesame and RQL to 
query over RDF instead of the approach in~\cite{Ejarque:2008:USR:1443230.1444322} where Jena is used. Their idea of 
running semantics in a federated environment is powerful idea but although the 
use of RDF could be solved some of the interoperability issues it is still under 
study.


In~\cite{Buyya:2010:IUF:2143583.2143586} a framework called Reference Architecture for Semantically Inter-operable 
Clouds (RASIC) is presented to facilitate the management of inter-cloud 
components and to provide reliable end to end services that meet the SLA 
requirements. This work tries to capture the concepts and attributes of 
resources in a cloud environment using semantics to address the problem of 
semantic interoperability between heterogeneous cooperating clouds.


A cloud computing ontology is proposed in~\cite{secloud12} to ease the semantic 
identification, discovery and access to cloud services Moreover, a tool that 
semi-automatically annotates cloud services and stores their semantic 
description is also introduced. Authors create ontologies and taxonomies trying 
to capture existing concepts and relationships in a cloud environment. 
Basically, they are focus on service discovery and selection according to 
functional and non-functional properties. After that they use a semantic 
description to describe cloud services and enable automatically the discovery of 
services using a semantic reasoner.

A semantic-based monitoring and management system (SAMM) is presented in~\cite{fg-2266}. 
This system shows a novel approach to automatic infrastructure scaling, based on 
the observation of business-related metrics with the objective of offering 
on-demand resource provisioning capabilities and high flexibility to manage 
cloud systems. By using ontologies to describe resources and metrics available 
for observation, SAMM has capabilities to express different system architectures 
and monitoring facilities. Owing to its module-based architecture based on OSGi 
bundles and services, it can be extended to support new technologies without 
much effort. Finally, to meet the requirement of being able to define rules in a 
convenient way, authors use a decision-making module based on the Esper~\cite{esper-project} event 
processing engine. The main outcome of this work is the possibilities of 
dynamically increasing the amount of resources taking into account both business 
and technical issues. In order to add flexibility to this system they use 
ontologies, rules and an event processing engine. They also suggest as future 
work the creation of a web interface to manage the functionalities and support 
other cloud stacks such as OpenStack or OpenNebula to make their tool 
inter-operable in a heterogeneous environments.


In~\cite{srt-15} authors present QoSMONaaS (Quality of Service MONitoring as a Service), 
a QoS monitoring facility built on top of the SRT-15, a cloud-oriented and 
CEP-based platform being developed in the context of the homonymous EU funded 
project. In particular they present the main components of QoSMONaaS (a semantic 
model, a SLA analyzer, a KPI Meter, a Breach Detector and a Violation Certifier) 
and illustrate QoSMONaaS operation and internals with respect to a substantial 
case study of an Internet of Thing application. In conclusion, the work address 
the implementation of a new monitoring tool in the context of the SRT-15 project 
which major contribution is an innovative approach to QoS monitoring based on 
complex event processing and content based routing (CBR). Finally, authors point 
out the possibility of combining statistical and logical reasoning to make 
predictions in a QoS aware cloud environment.


Author presents in~\cite{anastasi2011} a three-year research about QoS in Service Oriented 
Architectures in which his main contributions~\cite{cucinottaSOA_IA09,Konstanteli:2009:RGF:1632706.1633120,DBLP:conf/compsac/CucinottaAA09} consists in: 1) 
design and development of a real-time SOA with QoS negotiation and management 
capabilities, more specifically an effective way to guarantee QoS in service 
provisioning has been proposed by achieving temporal isolation between 
high-level software infrastructures and low-level control logic, exploiting a 
modified Linux kernel supporting real-time scheduling strategies. SLAs have been 
also extended in order to support QoS attributes related to individual 
activities in real-time; 2) design and development of a QoS registry for 
supporting the QoS management of adaptive service-oriented real-time 
applications that gathers QoS data to different functional behaviors of the 
application (application modes) and to predict the future performance based on 
data already collected in the past. This approach permits the correct resource 
allocation and self-configuration while providing QoS guarantees; 3) a 
methodology to support QoS management for virtualized services deployed in 
Service Oriented Infrastructures (SOIs). In particular, author proposes 
admission control policies for providing both deterministic and probabilistic 
guarantees for service activations within a predefined time frame and 4) design 
and development of a service-oriented, flexible and adaptable middleware for QoS 
configuration and management of Wireless Sensor Networks (WSNs). It is based on 
a contract negotiation scheme using SLAs and enables applications to reconfigure 
and maintain the network during its lifetime independently of the underlying WSN 
technology. Although this work perfectly address some of the challenges in a 
cloud environment it is mainly focused on the intermixing of real-time 
techniques with SOAs, whilst other aspects typical of SOA-based approaches to 
software design such as semantics are not provided.


In~\cite{DBLP:conf/compsac/CucinottaAA10} authors discuss the challenging problem of how to ensure predictable 
levels of QoS in cloud applications across multiple layers of the typical cloud 
infrastructure and how SLAs management and enforcement policy might look like. 
They present their advances on the experience developing the IRMOS platform 
(real-time cloud computing infrastructure in the context of the IRMOS European 
Project).The main outcomes of this work lies in the advance in the 
state-of-the-art in SLAs and in particular the expression of requirements in the 
language of the application domain, the user’s  needs are dynamically translated 
to infrastructure requirements in fine grained SLAs and a real-time method to 
evaluate and mitigate violations in SLAs. Thus, this framework adds to the 
already known benefits of cloud systems the possibility of executing interactive 
and resource-demanding applications with QoS guarantee. Although it is a 
promising platform there is no information on how this platform models the SLAs, 
resources or the QoS features.


Authors present in~\cite{Mabrouk:2009:SEQ:1564601.1564724} a QoS model to provide the appropriate ground for QoS 
engineering in Service Oriented Computing (SOC). The model is focused on 
emerging QoS features related to the dynamics of service environments such as 
user mobility and context awareness of application services. QoS is considered 
as an end-to-end basis by covering QoS features of all resources and actors 
involved in service provisioning such as network, device, service or end-user. 
The use of semantics appears to represent and enrich QoS features making use of 
the Web Service Quality Model (WSQM). Additionally this model is supposed to be 
extensible in specific domains where QoS factors are required. The main focus of 
this approach is the semantic representation through ontologies rather than 
specifying a new vocabulary for QoS. Similar to this approach appears other 
old-QoS models Maximilien~\cite{Maximilien:2004:FOD:1024866.1025003}, MOQ~\cite{Kim:2007:MWS:1359823.1359827}, 
QoSOnt~\cite{Dobson:2005:QQO:1090946.1091252}, Papaioannou~\cite{Papaioannou:2006:QOL:1129027.1129054}, Dobson~\cite{Dobson:2006:TUQ:1173701.1174285}, 
Marchetti~\cite{Marchetti:2004:QMM:1013367.1013377} or WiQoSM~\cite{Resta:2008:WIQ:1340085.1340215} that partially support issues and criteria 
such as semantic description of QoS features, device mobility and capabilities, 
network performance,  environment characteristics, adaptation, context 
awareness, user requirements among others. In most of cases these models are not 
up-to-date and are more focused on web services than QoS in cloud systems.


In~\cite{DBLP:conf/woa/Talia11} author makes a review of the marriage between clouds and agents 
discussing how this can done and which scientific areas and issues must be 
involved to carry out research work for producing intelligent cloud services. 
The main focus of this article is the convergence between multi-agent systems 
that need a reliable infrastructure and cloud computing that needs intelligent 
software with dynamic, flexible and autonomous behavior. This is a position 
paper that addresses the needs and requirements of agents running on the cloud 
but without any emphasis in QoS or semantics.


A study of the semantic technologies for enterprise cloud management is presented in~\cite{Haase:2010:STE:1940334.1940342}. Authors present the suite eCloudManager to address 
the topics of data integration, collaborative documentation and annotation, intelligent information access and analytics. One of the main conclusions is that a RDF approach can 
improve data integration in highly heterogeneous and changing enterprise environments and they also point out some remain challenges such as policies, reasoning, and complex event processing.


Following the same approach in~\cite{Mabrouk:2009:SEQ:1564601.1564724} authors 
present in~\cite{Mabrouk:2009:QSC:1656980.1656990} a QoS-aware service 
composition that enables the fulfilling of complex user tasks while meeting QoS 
constraints. One challenging issue in this topic is the selection of the best 
set of services to compose and meeting global QoS constraints defined by the 
user. This problem is considered to be a NP-hard problem. Furthermore in a 
dynamic environment the resolution of this problem becomes more complex. The 
main outcome of this work is an algorithm guided by a heuristic that provides 
the appropriate ground for QoS composition in dynamic service environments. 
Finally an evaluation of the efficiency of the algorithm is also presented. From 
a semantics point of view there is no any relevant advance and it is just a 
matchmaking service for discovering and selecting services according to a set of 
QoS constraints.

mOSAIC~\cite{Cretella:2012:UMS:2428736.2428805} is a an Open-Source API and 
Platform for Multiple Clouds that uses ontologies and semantics for providing a 
unified description of cloud components, interfaces, SLAs, QoS, APIs and 
requirements. The main objective is to enable a semantic framework in which 
reasoning processes can be carried out as well as SPARQL queries for 
discovering, selecting and matchmaking services. The use of semantics is based 
on OWL-S for describing services. 

The Q-Clouds system~\cite{Nathuji:2010:QMP:1755913.1755938} is a QoS-aware 
control framework that tunes resource allocations to mitigate performance 
interference effects. It uses on-line control feedback to build a multi-input 
multi-output (MIMO) model to capture performance interference interactions and 
it reuses this information to perform closed loop resource management. Authors 
also apply this functionality to allow applications to specify multiple levels 
of QoS. For such applications, Q-Clouds dynamically provisions underutilized 
resources to enable elevated QoS levels, thereby improving system efficiency. 
The main contribution of this work is a system to provide assurances in 
performance issues applying a MIMO model for capturing interference effects and 
driving a closed loop resource management controller.

Authors in~\cite{DBLP:journals/ijguc/Chang11} introduce a trust model for 
efficient reconfiguration and allocation of computing resources satisfying user 
requests. Their model collects and analyzes reliability based on historical 
information of servers in a cloud data center with the objective of providing a 
trust model for managing resources efficiently in cloud providers. Thus a 
reliable model is provided and validated against different datasets but no 
semantics is used in any process. 

QAComPS, a quality-aware federated computational semantic web service for 
computational  modelers, is presented in~\cite{dewqacomps} to provide a 
federated QABroker based on ontologies and making use of OWL2 features. 
Basically they perform a matchmaking reasoning process for discovering and 
selecting services according to a set of characteristics to meet user 
requirements. A SAWSDL interface is also used to transfer semantic annotation 
to/from the QAComPS service and QABroker. The final objective is to demonstrate 
a promising framework to make cloud computing more accessible and cost effective 
for computational modelers.

Authors present in~\cite{stantchev2009negotiating} an approach (a three-step method) 
to map SLAs and QoS requirements of business processes. They formalize the capabilities 
and requirements to finally compare them with the objective of detecting performance 
or reliability gaps. This method is evaluated as a dynamic technique to accommodate 
and improve the performance of individual services deployed in a grid or 
a cloud computing infrastructure. In~\cite{stantchev2011addressing} 
an application of a SLA management is proposed to address dependability in 
a SOA lifecycle. Authors describe the concepts and formalisms of each 
lifecycle stage (Model, Assemble, Deploy, and Manage). The final objective 
of this approach is to meet the user requirements offering optimized levels 
of dependability.

A taxonomy of QoS management and Service Selection Methodologies in cloud 
computing is presented in~\cite{qos-taxonomy}. This survey reviews the current 
status of QoS in web services and purpose a taxonomy to model the resources on 
cloud computing environments. The main objective of this work is to provide a 
taxonomy for service selection in grid computing, SOA and cloud computing as 
well as define the QoS characteristics such as user preferences, QoS source, 
context, web service attributes, semantic descriptions of web services, fuzzy 
preferences, roles, etc. Finally the taxonomy is tested using the Analytic 
Hierarchy Process (AHP) technique to make decisions about service selection.

In~\cite{DBLP:conf/ic/BernsteinV10} authors make a proposal for in inter-cloud 
exchanges using an ontology and XMPP for cataloging computing resources. They 
make queries via SPARQL to select the components of being exchanged. The main 
aim of this work is to provide a federated cloud environment but it is still an 
early stage of development.

ServiceRank~\cite{Wu:2009:CQS:1696051.1696105} is a new ranking method which 
considers quality of service aspects (such as response time and availability) as 
well as social perspectives of services (such as how they invoke each other via 
service composition). Authors present this new algorithm that has been 
implemented on SOAlive, a platform for creating and managing services and 
situational applications. The main outcome of this work is the combination of 
QoS metrics with social aspects but no semantics is applied in any of the 
process to select services. 

Other QoS ontology, onQoS, is presented in~\cite{Damiano:2009:OQL:1506129.1506143}. Authors make a study of the impact of 
Semantics for the management of QoS requirements in service-based 
applications and they also present the aforementioned ontology, its role for 
specifying service requirements and the onQoS-QL language to support queries for 
service discovery. Finally, in~\cite{Dautov:2013:ASC:2462307.2462312} authors present a proposal for an 
ontology-driven approach to self-management cloud application platforms using 
the MAPE-K reference model and another ontology-based framework for policy-driven 
governance in cloud application platforms is also presented 
in~\cite{DBLP:conf/icsoc/KourtesisP11}.

\subsection{Summary and Evaluation}
This section presents a summary/questionnaire of the most relevant aforementioned semantic approaches for QoS management 
with the aim of establishing an intituitive  but empirical comparison among the different QoS models. 
The evaluation of ``quality'' in ontologies is not a mere process as some works~\cite{DBLP:conf/dexa/dAquinSSS07,DBLP:conf/nldb/SabouFM09} 
have already demonstrated. That is why we summarize these models according to a set of features, see Table~\ref{features-qos-models}. 
This list is inspired in previous works but not exclusive in order to remark characteristics 
to take into account when we want to re-use or extend some of the existing QoS models. 
Each feature is evaluated following the next approaches:
\begin{itemize}
 \item Open ended questions using a word/sentence associated to the feature.
 \item Multiple choice using the Likert scale~\cite{albaum1997likert} value: 1-Strongly disagree, 2-Disagree, 3-Neither agree nor disagree, 4-Agree and 5-Strongly agree.
 \item Closed ended questions with a Yes (Y)/No (N) value. 
 \item Finally the symbol ``-'' is used to represent those unknown/missing/not applicable features.
\end{itemize}

\begin{table}[!ht]
\renewcommand{\arraystretch}{1.3}
\tiny
\begin{center}
\begin{tabular}[c]{|p{2.5cm}|p{5cm}|p{3cm}|} 
\hline
  \textbf{Feature} &  \textbf{Definition}  &  \textbf{Type} \\\hline
  Language & This feature indicates how data is modeled & Word/sentence associated \\ \hline
  Reasoning & The model enables some kind of reasoning process & Yes/No and Word/sentence associated  \\ \hline
  Accessibility & The model can be easily accessed in different context & Likert scale  \\ \hline
  Adaptability & The model can be easily configured to meet new requirements & Likert scale  \\ \hline
  Auditability & The model provides a mechanism to know how it is working & Likert scale  \\ \hline
  Extensibility & The model can be easily extended & Likert scale  \\ \hline
  Flexibility & The model can be configured on-demand adding/removing features & Likert scale  \\ \hline  
  Interoperability & The model can be integrated with third-parties  & Likert scale  \\ \hline
  Portability & The model can be easily move to different architectures & Likert scale  \\ \hline
  Usability & The model can be easily configured and exploited & Likert scale  \\ \hline
  Standards & The model is based in the (re) use of standards and vocabularies (compliance) & Likert scale\ \\ \hline
  Licensing & The type of license & Word/sentence associated \\ \hline
  Maturity & The model presents a good level of maturity, development or presence & Likert scale\\ \hline
  Update & The model is frequently updated & Likert scale\\ \hline
\hline
\end{tabular}
\caption{Features for selecting a semantic-based QoS model.}\label{features-qos-models}
  \end{center}
\end{table} 


\begin{sidewaystable}[!ht]
\renewcommand{\arraystretch}{1.3}
\tiny
\begin{center}
%\begin{tabular}[c]{|p{0.5cm}|p{0.6cm}|p{0.6cm}|p{0.6cm}|p{0.6cm}|p{0.8cm}|p{0.6cm}|p{0.6cm}|p{0.7cm}|p{0.5cm}|p{0.6cm}|p{0.6cm}|p{0.6cm}|p{0.6cm}|p{0.6cm}|p{0.6cm}|p{0.6cm}|p{0.6cm}|p{0.6cm}|p{0.6cm}|p{0.6cm}|p{0.8cm}|p{0.8cm}|}
\begin{tabular}[c]{|p{2.5cm}|p{1.2cm}|p{1.2cm}|p{1.2cm}|p{0.8cm}|p{0.8cm}|p{0.8cm}|p{0.8cm}|p{1.1cm}|p{1.1cm}|p{1.1cm}|p{1.1cm}|p{1.1cm}|p{1.1cm}|p{1.1cm}|} 
\hline
  \textbf{Model/ Feature} & \textbf{Language} & \textbf{Reasoning} & \textbf{Access.} & \textbf{Adapt.} & \textbf{Audit.} & \textbf{Extens.} & \textbf{Flex.} & \textbf{Interoper.} & \textbf{Port.} & \textbf{Usability} & \textbf{Standards} & \textbf{Licensing} & \textbf{Maturity} & \textbf{Update} \\ \hline
   SERA~\cite{Ejarque:2008:USR:1443230.1444322} & OWL & Y & 1 & 3 & 4 & 3 & 2 & 4 & 3 & 3 & 4 & - & 2 & 2 \\ \hline
   OReSS~\cite{2009gdc..conf..221Y} & OWL & Y & 1 & 2 & 3 & 4 & 4 & 2 & 2 & 3 & 4 & - & 2 & 2 \\ \hline
   SITIO~\cite{Garcia-Sanchez:2010:ASS:1852403.1852409} & OWL + Rules & Y & 1 & 4 & 4  & 3 & 4 & 3 & 4 & 4 & 4 & - & 2 & 2 \\ \hline
   Cloud Recommender~\cite{DBLP:conf/gecon/ZhangRNMH12} & - & 1 & 4 & 3 & 4 & 2 & 2 & 3 & 2 & 4 & 3 & - & 2 & 1 \\ \hline
   FP7 4WARD~\cite{5682131} & OWL & Y & 3 & 2 & 4 & 2 & 2 & 3 & 2& 3 & 3 & -& 1 & 1 \\ \hline
   SRC~\cite{DBLP:conf/soca/ChenL10} & OWL & Y & 1 & 3 & 3 & 4 & 2 & 2 &3 & 3  & 3 & - & 2 & 2 \\ \hline
   Cloudle~\cite{5682131} & OWL & Y & 1 & 2 & 4 & 3& 2& 2 & 3& 2& 3& -&2 &1 \\ \hline
   IRPS~\cite{6206823} & OWL + RDQL& Y & 1 & 4&3 &4 & 3& 4& 2& 3& 2& -&1&1 \\ \hline
   RASIC~\cite{Buyya:2010:IUF:2143583.2143586} &- & -& 1 & 4 & 4& 3& 4& 4& 4& 3& 3&-&1 &1 \\ \hline
   SAMM~\cite{fg-2266} & OWL + ESPER & Y & 1 & 4&4 &4 &4 &4 & 4& 3& 3&- & 1&1 \\ \hline
   QoSMONaaS~\cite{srt-15} & OWL + CEP + SLA& Y & 1&3 &4 &4 & 4& 4& 3& 4& 3& -& 1&1 \\ \hline
   IRMOS~\cite{DBLP:conf/compsac/CucinottaAA10}  & -&- &1 & 3& 4& 4& 4& 2& 2& 2& 3& -&2 &2 \\ \hline
   QoS model~\cite{Mabrouk:2009:SEQ:1564601.1564724} & WSQM & - &1 &4 &2 &4 &4 &4 &2 &2 &4 &- &1 &1 \\ \hline
   Maximilien~\cite{Maximilien:2004:FOD:1024866.1025003} & OWL & Y & 1&4 &3 &4 &4 &4 &4 &3 &3 &- &1 &1 \\ \hline
   MOQ~\cite{Kim:2007:MWS:1359823.1359827} & OWL & Y & 1&4 &3 &4 &4 &4 &4 &3 &3 &- &1 &1 \\ \hline
   Papaioannou\cite{Papaioannou:2006:QOL:1129027.1129054} & OWL & Y & 1&3 &3 &3 &3 &3 &3 &3 &3 &- &1 &1 \\ \hline
   Dobson~\cite{Dobson:2006:TUQ:1173701.1174285} & OWL & Y & 1&3 &4 &4 &4 &4 &3 &4 &4 &- &1 &1 \\ \hline
   Marchetti~\cite{Marchetti:2004:QMM:1013367.1013377} & - & - & 1 &2 &2 &2 &2 &2 &2 &3 &3 &- &1 &1 \\ \hline
   WiQoSM~\cite{Resta:2008:WIQ:1340085.1340215} & - & - & 1 &2 &2 &2 &2 &2 &2 &2 &2 &- &1 &1 \\ \hline
   QoSOnt~\cite{Dobson:2005:QQO:1090946.1091252} & OWL & Y & 1&  &3 &4 &4 &4 &4 &3 &3 &- &1 &1 \\ \hline
   eCloudManager~\cite{Haase:2010:STE:1940334.1940342} & RDF & Y & 1& 3 &3 &3 &3 &3 &3 &2 &4 &- &1 &1 \\ \hline
   mOSAIC~\cite{Cretella:2012:UMS:2428736.2428805} & OWL-S + SPARQL & Y & 1 & 4 &4 &4 &4 &3 &3 &4 &4 &- &1 &1 \\ \hline
   Q-Clouds~\cite{Nathuji:2010:QMP:1755913.1755938} & MIMO & - & 1 & 3 &3 &4 &4 &3 &3 &3 &3 &- &1 &1 \\ \hline
   QAComPS~\cite{dewqacomps} & OWL2 + SAWSDL & Y & 1 & 3 &3 &4 &4 &4 &4 &3 &4 &- &1 &1 \\ \hline
   ServiceRank~\cite{Wu:2009:CQS:1696051.1696105} & - & - & 1 & 3 &3 &2 &3 &34 &3 &2 &2 &- &2 &2 \\ \hline
   onQoS~\cite{Damiano:2009:OQL:1506129.1506143} & OWL + onQoS-QL & Y & 1 & 3 &3 &3 &4 &4 &4 &3 &2 &- &1 &1 \\ \hline
   QoS \& SLAs~\cite{stantchev2009negotiating} & - & - & 1 & 4 & 4 &4 &4 &4 &4 &4 &3 &- &2 &2 \\ \hline
   SOA \& Dependability~\cite{stantchev2011addressing} & - & - & 1 & 4 & 4 &4 &4 &4 &4 &4 &3 &- &2 &2 \\ \hline   
   QoS Taxonomy~\cite{qos-taxonomy} & - & - & 1 & 3 & 3 &3 &3 &3 &3 &3 &3 &- &1 &1 \\ \hline   
\hline
\end{tabular}
\caption{Summary of Ontology-based frameworks for QoS management.}\label{summary-features-qos-models}
  \end{center}
\end{sidewaystable} 

At a first glance and according to results in Table~\ref{summary-features-qos-models} it seems that most of the QoS models 
based on semantics are not publicly available. Although in most of them some uses of OWL or SPARQL are very promising 
to define profiles, restrictions, SLAs, ECA rules, etc. there is no way of re-using them avoiding one of the main 
principles of semantic web technologies. In most of cases they have been developed within the execution of some 
research project but, at the moment, are not up-to-date. Apart from that the number of QoS models implies that 
there is no consensus in building a common set of QoS indicators and, maybe, this is one of the reasons that 
prevents the real deployment of quality-based methods. Although the use of the Likert scale to evaluate 
some features presents some disadvantages it can serve as a guide to ranking different approaches. As a final remark, semantic 
technologies have been applied to QoS but violating some of the basic principles: 1) re-use of existing vocabularies and 2) 
build on the top of a common and shared understanding. Further steps in this area should be the creation of a common set 
of quality indicators and their representation using semantic web technologies with the aim of enabling a better 
re-use of knowledge and dissemination. Although it is not easy to reach an agreement in a broad field like QoS some 
minimal common definitions should be established. In this sense, the initiatives presented in Section~\ref{qos-cloud-index} 
can be a step-forward to boost the application of QoS in real environments.

\clearpage
