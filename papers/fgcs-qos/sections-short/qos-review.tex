In this section a literature review of main ontology-based frameworks for QoS management is presented. After that 
an empirical evaluation of some selected features, see Table~\ref{features-qos-models}, is also outlined to finally present 
a summary, see Table~\ref{summary-features-qos-models}, of the most relevant approaches for semantic-based QoS management.
% These issues are considered to be key-enablers of the future semantic-based QoS management in cloud environments.
\subsection{Ontology-based frameworks for QoS management}
An ontology-based resource description is proposed in~\cite{Pernas:2005:UOD:1068510.1069326,Armstrong17} to solve problems with regards to the difficulty of 
resource information management, no standard definitions of resource requirements and the difficulty of guaranteeing compatibility of resource allocation. 
In this sense, there are also works trying to produce a global ontology by merging existing ontologies of 
resource groups~\cite{Lopes:2006:PEM:1135771.1136110}. On the other hand, authors in~\cite{Ejarque:2008:USR:1443230.1444322} propose 
the Semantically-Enhanced Resource Allocator (SERA), a scheduling system using customer requests with the ability of re-scheduling 
requests based on their priorities and considering advanced reservations.

% In~\cite{2009gdc..conf..221Y} authors propose a resource virtualization method using a virtual ontology (VOn) and an execution environment called OReSS 
% (Ontology-based Resource Selection Service) that is configured dynamically based on user requirements. The VOn is mapped to cloud resources 
% and a Map/Reduce technique is applied for the rapid and efficient merging of a number of ontologies. The execution environment is comprised of resources 
% described using the VOn and they are automatically populated applying the Ontology Merge engine. The main contribution of this work lies in 
% the resource management in cloud computing systems tackling the complex management of distributed heterogeneous resources causes the aforementioned problems. 
% In order to solve these problems a resource virtualization method is proposed using ontologies in a cloud environment. 

In~\cite{rule-2013-resource-provisioning} a Rule Based Resource Manager is proposed for a cloud hybrid environment 
with the objective of increasing the scalability (on-demand) of private clouds. This work also sets up the execution time 
for public and private cloud in order to fulfill requests selected different services. In this case, the methodology is applied to 
the IaaS layer to access resources on demand enabling the scale up of private clouds with a cost-effective.

The SITIO platform~\cite{Garcia-Sanchez:2010:ASS:1852403.1852409} gathers the emerging concepts of SaaS, semantic technologies, Business Process 
Modeling and Cloud Computing to foster the dramatic evolution of a new platform oriented towards interoperability and cost reduction. SITIO is defined as a 
platform for reliable, privacy-aware, secure and cost-efficient semantics-based Software-as-a-Service Creation, Integration and Management. 
Its main contribution lies in a methodology for annotating services using old-fashioned semantic web services techniques.  Although the SITIO platform 
applies semantics to solve interoperability problems it is only focused on the description of web services functionality and capabilities.

A declarative system called CloudRecommender is presented in~\cite{DBLP:conf/gecon/ZhangRNMH12} 
through an unified and formalized domain model capable of describing infrastructure 
services such as Amazon, Microsoft Azure, GoGrid, etc. A prototype is also introduced to 
show the main benefits of this approach: 1) a recommender with the capability of estimating costs across 
multiple providers, 2) the aid in the selection of cloud services and 3) an user-friendly service interface based on 
widgets that maps user requirements (form inputs) to available infrastructure services. As future work 
authors suggest the use of the recommender to support the selection of more cloud service types 
such as PaaS services including run-time features.


In~\cite{5682131} authors review three concepts developed in the context of 
the FP7 4WARD project with the objective of demonstrating their potential impact on QoS 
management: network virtualization (to decouple network from infrastructure and overcome ``one-size-fits-all''), 
generic path semantic resource management (a QoS profile to overcome inadequacies of the traditional layered network model) 
and in-network management (to incorporate QoS management capabilities in network elements). All of them 
are novel and promising concepts that are being targeted at handling QoS issues and are supposed to be 
relevant for enabling a new dynamic, flexible, adaptable and scalable cloud environment.

Authors in~\cite{DBLP:conf/soca/ChenL10} aim to provide a suitable service cater to discover consumer 
service requests including functional requirements and non-functional 
properties. They propose a service registry model named ``SRC'' which is an 
extension of a keyword based service registry model. The SRC is deployed as a 
cloud application to provide a behavior-aware and QoS aware service discovery 
storing both semantic descriptors of web services and QoS feedback. This data is processed using a Map/Reduce 
mechanism. Basically it is a matchmaking service based on the WSDL descriptions 
taking advantage of using OWL for simple annotations of functional and 
non-functional properties. The main drawback of this approach 
lies in the necessity of ensuring synchronized multiple copies of OWL definitions on all nodes.


In~\cite{cloudle} a search engine and an architecture for cloud systems (Cloudle) is 
outlined to semantically look up services according to a user profile. Two ontologies (T-Box and A-Box) 
have been also designed in order to assist this similarity reasoning process and are used to 
improve the accuracy of results. The main finding of this study is that an enriched 
ontology can improve the selection of cloud services. However all concepts, 
properties, etc. are defined by the authors from the scratch without any reuse 
of existing standard concepts.

In~\cite{6206823} authors introduce the system Inter-cloud Resource Provisioning System 
(IRPS) to accomplish the requirements of a customer providing additional resources in a federated cloud system. 
This system schedules some tasks to allocate resources by using semantics and an inference engine; 
more specifically they use Sesame and RQL to query over RDF instead of the approach in~\cite{Ejarque:2008:USR:1443230.1444322} 
where Jena is used. Although the approach of running semantics in a federated environment 
is powerful and the use of RDF can solved interoperability problems, the distributed 
execution of queries is still under study.


In~\cite{Buyya:2010:IUF:2143583.2143586} a framework called 
Reference Architecture for Semantically Inter-operable Clouds (RASIC) is presented to facilitate the management of inter-cloud 
components and to provide reliable end to end services that meet SLA requirements. 
This work tries to capture the concepts and attributes of resources in a cloud 
environment using semantics to address the problem of semantic interoperability between heterogeneous cooperating clouds.


A cloud computing ontology is proposed in~\cite{secloud12} to ease the semantic 
identification, discovery and access to cloud services. Authors create ontologies 
and taxonomies trying to capture existing concepts and relationships in a cloud environment. 
Basically, they are focused on service discovery and selection according to 
functional and non-functional properties.

A semantic-based monitoring and management system (SAMM) is presented in~\cite{fg-2266}. 
This system shows a novel approach to automatic infrastructure scaling, based on 
the observation of business-related metrics with the objective of offering 
on-demand resource provisioning capabilities and high flexibility to manage 
cloud systems. By using ontologies to describe resources and metrics available 
for observation, SAMM provides capabilities to express different system architectures 
and monitoring facilities. The architecture is based on OSGi including a decision-making module based 
on the Esper event processing engine. The main outcome of this work lies in the possibility of 
dynamically increasing the amount of resources taking into account both business 
and technical issues. This tool is also supposed to support cloud stacks such 
as OpenStack or OpenNebula.

In~\cite{srt-15} authors present QoSMONaaS (Quality of Service MONitoring as a Service), 
a QoS monitoring facility built on top of the SRT-15 platform (a cloud-oriented and CEP-based system). 
In particular they present the main components of QoSMONaaS: 1) a semantic model; 2) a SLA analyzer; 3) a KPI Meter; 4) a Breach Detector and 
5) a Violation Certifier. This work addresses the monitoring problem using 
a CEP-based approach. Finally, authors also point out the possibility of combining statistical and 
logical reasoning to make predictions in a QoS aware cloud environment.


Author presents in~\cite{anastasi2011} a three-year research about QoS in Service Oriented Architectures 
in which his main contributions~\cite{cucinottaSOA_IA09,Konstanteli:2009:RGF:1632706.1633120,DBLP:conf/compsac/CucinottaAA09} 
consists in: 1) design and development of a real-time SOA with QoS negotiation and management 
capabilities; 2) design and development of a QoS registry to support the QoS management of 
adaptive service-oriented real-time applications including functional behaviors and to predict 
the future performance based on data already collected; 3) a methodology to support 
QoS management for virtualized services deployed in Service Oriented Infrastructures (SOIs) and 
4) design and development of a service-oriented, flexible and adaptable middleware for QoS 
configuration and management of Wireless Sensor Networks (WSNs). Although this work perfectly tackles 
some of the challenges in a cloud environment it is mainly focused on the intermixing of real-time 
techniques with SOAs, whilst other aspects typical of SOA-based approaches to 
software design such as semantics are not provided.


In~\cite{DBLP:conf/compsac/CucinottaAA10} authors present their experiences while developing the IRMOS platform (a real-time cloud computing infrastructure developed in the context of the IRMOS European Project). 
The main outcome of this work lies in the advance in the state-of-the-art in SLAs and, in particular, the expression of requirements 
in the language of the application domain: user's  needs are dynamically translated to infrastructure requirements in fine 
grained SLAs and a real-time method has been designed to evaluate and mitigate violations in SLAs. Although it is a 
promising platform there is no information about how this platform models the SLAs, resources or the QoS features.


Authors introduce in~\cite{Mabrouk:2009:SEQ:1564601.1564724} a QoS model to provide the appropriate ground for QoS 
engineering in Service Oriented Computing (SOC). The model is focused on 
emerging QoS features related to the dynamics of service environments such as 
user mobility and context of application services. In this case, the use of semantics emerges 
to represent and enrich QoS features making use of the Web Service Quality Model (WSQM).

% Additionally this model is supposed to be extensible in specific domains where QoS factors are required. The main focus of 
% this approach is the semantic representation through ontologies rather than 
% specifying a new vocabulary for QoS. Similar to this approach appears other 
% old-QoS models Maximilien, MOQ, QoSOnt, Papaioannou, Dobson, Marchetti or WiQoSM that partially support issues and criteria 
% such as semantic description of QoS features, device mobility and capabilities, 
% network performance,  environment characteristics, adaptation, context 
% awareness, user requirements among others. In most of cases these models are not 
% up-to-date and are more focused on web services than QoS in cloud systems.


In~\cite{DBLP:conf/woa/Talia11} author makes a review of the marriage between clouds and agents 
discussing how this can be done and which scientific areas and issues must be 
involved to carry out research works for producing intelligent cloud services. 
The main focus of this article is the convergence between multi-agent systems 
that need a reliable infrastructure and cloud computing that needs intelligent 
software with dynamic, flexible and autonomous behavior. 


A study of the semantic technologies for enterprise cloud management is presented in~\cite{Haase:2010:STE:1940334.1940342}. 
Authors present the suite eCloudManager to address the topics of data integration, collaborative documentation and annotation, 
intelligent information access and analytics. One of the main conclusions is that a RDF approach can improve data integration 
in highly heterogeneous and changing enterprise environments in which complex event processing and reasoning can be key-processes to 
enable a smart environment.


Following a similar approach to~\cite{Mabrouk:2009:SEQ:1564601.1564724} authors 
present in~\cite{Mabrouk:2009:QSC:1656980.1656990} a QoS-aware service 
composition that enables the fulfilling of complex user tasks while meeting QoS 
constraints. One challenging issue in this topic is the selection of the best 
set of services (NP-hard problem) to compose and meeting global QoS constraints defined by the 
user. The main outcome of this work is an algorithm guided by a heuristic that provides 
the appropriate ground for QoS composition in dynamic service environments. 

The mOSAIC~\cite{Cretella:2012:UMS:2428736.2428805} platform for multiple clouds uses ontologies and 
semantics for providing a unified description of cloud components, interfaces, SLAs, QoS, APIs and requirements. 
The main objective of this platform is to enable a semantic framework in which reasoning processes can be carried out as well as 
SPARQL queries for discovering, selecting and matchmaking services. Semantic technologies are applied to describe services using OWL-S. 

The Q-Clouds system~\cite{Nathuji:2010:QMP:1755913.1755938} is a QoS-aware 
control framework that tunes (applying an on-line control feedback to build a  MIMO-multi-input multi-output model) 
resource allocations to mitigate performance interference effects. The main contribution of this work is a system to 
provide assurance in performance issues applying a MIMO model for capturing interference effects and driving 
a closed loop resource management controller.

Authors in~\cite{DBLP:journals/ijguc/Chang11} introduce a trust model for 
efficient reconfiguration and allocation of computing resources satisfying user 
requests. This model collects and analyzes reliability based on historical information gathered from a cloud data center. 
Thus the model is provided and validated against different datasets but no semantics is used in any process. 

QAComPS, a quality-aware federated computational semantic web service for 
computational  modelers, is presented in~\cite{dewqacomps} to provide a 
federated QABroker based on ontologies and making use of OWL2 features. 
Basically they perform a matchmaking reasoning process for discovering and 
selecting services according to a set of characteristics to meet user 
requirements. A SAWSDL interface is also published to transfer semantic annotation 
to/from the QAComPS service and QABroker.

Authors present in~\cite{stantchev2009negotiating} an approach (a three-step method) 
to map SLAs and QoS requirements of business processes. They formalize the capabilities 
and requirements to finally compare them with the objective of detecting performance 
or reliability gaps. This method is evaluated as a dynamic technique to accommodate 
and improve the performance of individual services deployed in a grid or 
a cloud computing infrastructure. In~\cite{stantchev2011addressing} 
an application of a SLA management is proposed to address dependability in 
a SOA lifecycle. Authors describe the concepts and formalisms of each 
lifecycle stage (Model, Assemble, Deploy, and Manage). The final objective 
of this approach is to meet user requirements offering optimized levels 
of dependability.

A taxonomy of QoS management and Service Selection Methodologies in cloud 
computing is presented in~\cite{qos-taxonomy}. This survey reviews the current 
status of QoS in web services and purposes a taxonomy to model the resources on 
cloud computing environments. The main objective of this work is to provide a 
taxonomy for service selection in grid computing, SOA and cloud computing as 
well as define QoS characteristics such as user preferences, QoS source, 
context, web service attributes, semantic descriptions of web services, fuzzy 
preferences, roles, etc. Finally the taxonomy is tested using the Analytic 
Hierarchy Process (AHP) technique to make decisions about service selection.

In~\cite{DBLP:conf/ic/BernsteinV10} authors make a proposal for inter-cloud 
exchanges (XMPP) and cataloging of computing resources (ontology). They 
perform queries via a SPARQL endpoint to select the components to be exchanged. The main 
aim of this work is to provide a federated cloud environment but it is still an 
early stage of development.

ServiceRank~\cite{Wu:2009:CQS:1696051.1696105} is a new ranking method which 
considers quality of service aspects (such as response time and availability) as 
well as social perspectives of services (such as how they invoke each other via 
service composition). Authors present this new algorithm that has been 
implemented on SOAlive, a platform for creating and managing services and 
situational applications. The main outcome of this work is the combination of 
QoS metrics with social aspects but no semantics is applied in any of the 
process to select services. 

Other QoS ontology, onQoS, is presented in~\cite{Damiano:2009:OQL:1506129.1506143}. 
Authors make a study of the impact of Semantics for the management of QoS requirements in service-based 
applications and they also present the aforementioned ontology, its role for 
specifying service requirements and the onQoS-QL language to support queries for 
service discovery. Finally, in~\cite{Dautov:2013:ASC:2462307.2462312} authors present a proposal for an 
ontology-driven approach to self-management of cloud application platforms using 
the MAPE-K reference model and another ontology-based framework for policy-driven 
governance in cloud application platforms is also presented in~\cite{DBLP:conf/icsoc/KourtesisP11}.

\subsection{Summary and Evaluation}
This section presents a summary/questionnaire of the most relevant aforementioned semantic approaches for QoS management 
with the aim of establishing an intuitive  but empirical comparison among the different QoS models. 
The evaluation of ``quality'' in ontologies is not a mere process as some works~\cite{DBLP:conf/dexa/dAquinSSS07,DBLP:conf/nldb/SabouFM09} 
have already demonstrated. That is why we summarize these models according to a set of features, see Table~\ref{features-qos-models}. 
This list is inspired in previous works but not exclusive in order to remark characteristics to take into account when 
we want to re-use or extend some of the existing QoS models. Each feature is evaluated following the next approaches:
\begin{itemize}
 \item Open ended questions using a word/sentence associated to the feature.
 \item Multiple choice using the Likert scale~\cite{albaum1997likert} value: 1-Strongly disagree, 2-Disagree, 3-Neither agree nor disagree, 4-Agree and 5-Strongly agree.
 \item Closed ended questions with a Yes (Y)/No (N) value. 
 \item Finally the symbol ``-'' is used to represent those unknown/missing/not applicable features.
\end{itemize}

\begin{table}[!ht]
\renewcommand{\arraystretch}{1.3}
\tiny
\begin{center}
\begin{tabular}[c]{|p{2.5cm}|p{5cm}|p{3cm}|} 
\hline
  \textbf{Feature} &  \textbf{Definition}  &  \textbf{Type} \\\hline
  Language & This feature indicates how data is modeled & Word/sentence associated \\ \hline
  Reasoning & The model enables some kind of reasoning process & Yes/No and Word/sentence associated  \\ \hline
  Accessibility & The model can be easily accessed in different context & Likert scale  \\ \hline
  Adaptability & The model can be easily configured to meet new requirements & Likert scale  \\ \hline
  Auditability & The model provides a mechanism to know how it is working & Likert scale  \\ \hline
  Extensibility & The model can be easily extended & Likert scale  \\ \hline
  Flexibility & The model can be configured on-demand adding/removing features & Likert scale  \\ \hline  
  Interoperability & The model can be integrated with third-parties  & Likert scale  \\ \hline
  Portability & The model can be easily move to different architectures & Likert scale  \\ \hline
  Usability & The model can be easily configured and exploited & Likert scale  \\ \hline
  Standards & The model is based in the (re) use of standards and vocabularies (compliance) & Likert scale\ \\ \hline
  Licensing & The type of license & Word/sentence associated \\ \hline
  Maturity & The model presents a good level of maturity, development or presence & Likert scale\\ \hline
  Update & The model is frequently updated & Likert scale\\ \hline
\hline
\end{tabular}
\caption{Features for selecting a semantic-based QoS model.}\label{features-qos-models}
  \end{center}
\end{table} 


\begin{sidewaystable}[!ht]
\renewcommand{\arraystretch}{1.3}
\tiny
\begin{center}
%\begin{tabular}[c]{|p{0.5cm}|p{0.6cm}|p{0.6cm}|p{0.6cm}|p{0.6cm}|p{0.8cm}|p{0.6cm}|p{0.6cm}|p{0.7cm}|p{0.5cm}|p{0.6cm}|p{0.6cm}|p{0.6cm}|p{0.6cm}|p{0.6cm}|p{0.6cm}|p{0.6cm}|p{0.6cm}|p{0.6cm}|p{0.6cm}|p{0.6cm}|p{0.8cm}|p{0.8cm}|}
\begin{tabular}[c]{|p{2.5cm}|p{1.2cm}|p{1.2cm}|p{1.2cm}|p{0.8cm}|p{0.8cm}|p{0.8cm}|p{0.8cm}|p{1.1cm}|p{1.1cm}|p{1.1cm}|p{1.1cm}|p{1.1cm}|p{1.1cm}|p{1.1cm}|} 
\hline
  \textbf{Model/ Feature} & \textbf{Language} & \textbf{Reasoning} & \textbf{Access.} & \textbf{Adapt.} & \textbf{Audit.} & \textbf{Extens.} & \textbf{Flex.} & \textbf{Interoper.} & \textbf{Port.} & \textbf{Usability} & \textbf{Standards} & \textbf{Licensing} & \textbf{Maturity} & \textbf{Update} \\ \hline
   SERA~\cite{Ejarque:2008:USR:1443230.1444322} & OWL & Y & 1 & 3 & 4 & 3 & 2 & 4 & 3 & 3 & 4 & - & 2 & 2 \\ \hline
   OReSS~\cite{2009gdc..conf..221Y} & OWL & Y & 1 & 2 & 3 & 4 & 4 & 2 & 2 & 3 & 4 & - & 2 & 2 \\ \hline
   SITIO~\cite{Garcia-Sanchez:2010:ASS:1852403.1852409} & OWL + Rules & Y & 1 & 4 & 4  & 3 & 4 & 3 & 4 & 4 & 4 & - & 2 & 2 \\ \hline
   Cloud Recommender~\cite{DBLP:conf/gecon/ZhangRNMH12} & - & 1 & 4 & 3 & 4 & 2 & 2 & 3 & 2 & 4 & 3 & - & 2 & 1 \\ \hline
   FP7 4WARD~\cite{5682131} & OWL & Y & 3 & 2 & 4 & 2 & 2 & 3 & 2& 3 & 3 & -& 1 & 1 \\ \hline
   SRC~\cite{DBLP:conf/soca/ChenL10} & OWL & Y & 1 & 3 & 3 & 4 & 2 & 2 &3 & 3  & 3 & - & 2 & 2 \\ \hline
   Cloudle~\cite{5682131} & OWL & Y & 1 & 2 & 4 & 3& 2& 2 & 3& 2& 3& -&2 &1 \\ \hline
   IRPS~\cite{6206823} & OWL + RDQL& Y & 1 & 4&3 &4 & 3& 4& 2& 3& 2& -&1&1 \\ \hline
   RASIC~\cite{Buyya:2010:IUF:2143583.2143586} &- & -& 1 & 4 & 4& 3& 4& 4& 4& 3& 3&-&1 &1 \\ \hline
   SAMM~\cite{fg-2266} & OWL + ESPER & Y & 1 & 4&4 &4 &4 &4 & 4& 3& 3&- & 1&1 \\ \hline
   QoSMONaaS~\cite{srt-15} & OWL + CEP + SLA& Y & 1&3 &4 &4 & 4& 4& 3& 4& 3& -& 1&1 \\ \hline
   IRMOS~\cite{DBLP:conf/compsac/CucinottaAA10}  & -&- &1 & 3& 4& 4& 4& 2& 2& 2& 3& -&2 &2 \\ \hline
   QoS model~\cite{Mabrouk:2009:SEQ:1564601.1564724} & WSQM & - &1 &4 &2 &4 &4 &4 &2 &2 &4 &- &1 &1 \\ \hline
 %  Maximilien~\cite{Maximilien:2004:FOD:1024866.1025003} & OWL & Y & 1&4 &3 &4 &4 &4 &4 &3 &3 &- &1 &1 \\ \hline
 %  MOQ~\cite{Kim:2007:MWS:1359823.1359827} & OWL & Y & 1&4 &3 &4 &4 &4 &4 &3 &3 &- &1 &1 \\ \hline
 %  Papaioannou\cite{Papaioannou:2006:QOL:1129027.1129054} & OWL & Y & 1&3 &3 &3 &3 &3 &3 &3 &3 &- &1 &1 \\ \hline
 %  Dobson~\cite{Dobson:2006:TUQ:1173701.1174285} & OWL & Y & 1&3 &4 &4 &4 &4 &3 &4 &4 &- &1 &1 \\ \hline
 %  Marchetti~\cite{Marchetti:2004:QMM:1013367.1013377} & - & - & 1 &2 &2 &2 &2 &2 &2 &3 &3 &- &1 &1 \\ \hline
 %  WiQoSM~\cite{Resta:2008:WIQ:1340085.1340215} & - & - & 1 &2 &2 &2 &2 &2 &2 &2 &2 &- &1 &1 \\ \hline
 %  QoSOnt~\cite{Dobson:2005:QQO:1090946.1091252} & OWL & Y & 1&  &3 &4 &4 &4 &4 &3 &3 &- &1 &1 \\ \hline
   eCloudManager~\cite{Haase:2010:STE:1940334.1940342} & RDF & Y & 1& 3 &3 &3 &3 &3 &3 &2 &4 &- &1 &1 \\ \hline
   mOSAIC~\cite{Cretella:2012:UMS:2428736.2428805} & OWL-S + SPARQL & Y & 1 & 4 &4 &4 &4 &3 &3 &4 &4 &- &1 &1 \\ \hline
   Q-Clouds~\cite{Nathuji:2010:QMP:1755913.1755938} & MIMO & - & 1 & 3 &3 &4 &4 &3 &3 &3 &3 &- &1 &1 \\ \hline
   QAComPS~\cite{dewqacomps} & OWL2 + SAWSDL & Y & 1 & 3 &3 &4 &4 &4 &4 &3 &4 &- &1 &1 \\ \hline
   ServiceRank~\cite{Wu:2009:CQS:1696051.1696105} & - & - & 1 & 3 &3 &2 &3 &34 &3 &2 &2 &- &2 &2 \\ \hline
   onQoS~\cite{Damiano:2009:OQL:1506129.1506143} & OWL + onQoS-QL & Y & 1 & 3 &3 &3 &4 &4 &4 &3 &2 &- &1 &1 \\ \hline
   QoS \& SLAs~\cite{stantchev2009negotiating} & - & - & 1 & 4 & 4 &4 &4 &4 &4 &4 &3 &- &2 &2 \\ \hline
   SOA \& Dependability~\cite{stantchev2011addressing} & - & - & 1 & 4 & 4 &4 &4 &4 &4 &4 &3 &- &2 &2 \\ \hline   
   QoS Taxonomy~\cite{qos-taxonomy} & - & - & 1 & 3 & 3 &3 &3 &3 &3 &3 &3 &- &1 &1 \\ \hline   
\hline
\end{tabular}
\caption{Summary of Ontology-based frameworks for QoS management.}\label{summary-features-qos-models}
  \end{center}
\end{sidewaystable} 

At a first glance and according to results in Table~\ref{summary-features-qos-models} it seems that most of the QoS models 
based on semantics are not publicly available. Although in most of them some uses of OWL or SPARQL are very promising 
to define profiles, restrictions, SLAs, ECA rules, etc. there is no way of re-using them avoiding one of the main 
principles of semantic web technologies. In most of cases they have been developed within the execution of some 
research project but, at the moment, are not up-to-date. Apart from that the number of QoS models implies that 
there is no consensus in building a common set of QoS indicators and, maybe, this is one of the reasons that 
prevents the real deployment of quality-based methods. On the other hand, the use of the Likert scale to evaluate 
some features presents some disadvantages it can serve as a guide to ranking different approaches. As a final remark, semantic 
technologies have been applied to QoS but violating some of the basic principles: 1) re-use of existing vocabularies and 2) 
build on the top of a common and shared understanding. Further steps in this area should be the creation of a common set 
of quality indicators and their representation using semantic web technologies with the aim of enabling a better 
re-use of knowledge and dissemination. Although it is not easy to reach an agreement in a broad field like QoS some 
minimal common definitions should be established. In this sense, the initiatives presented in Section~\ref{qos-cloud-index} 
can be a step-forward to boost the application of QoS in real environments.

\clearpage
